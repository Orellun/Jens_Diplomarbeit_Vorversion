\chapter*{Zusammenfassung}\addcontentsline{toc}{chapter}{Zusammenfassung}


Die Firma Bihler entwickelt und produziert für ihre Stanz-Biegeautomaten NC-Aggregate. Mit diesen werden Werkzeuge präzise bedient, um Bauteile und Baugruppen in der Massenproduktion zu fertigen. Um zu vermeiden, dass es zu Funktionsfehlern bei den in Serie gefertigten NCAs kommt, wird durch Prüfen während des Produktionsprozesses versucht, derartige Probleme so weit wie möglich zu auszuschließen. Obwohl die Aggregate bereits ein Prüfverfahren durchlaufen, werden nicht alle gravierenden Fehler entdeckt. Deshalb wird das derzeitige Prüfen analysiert und es werden Vorschläge zu einer eventuell nötigen Überarbeitung entwickelt.


Hierzu wird der Aufbau der verschiedenen Varianten der NCAs dargestellt, die elektronische Ansteuerung unter Zuhilfenahme des Kurvengetriebes erläutert und die Fahrbewegung der NCAs im Trapez- und Dreiecksprofil aufgezeigt. Es wird festgestellt, dass Funktionsstörungen der NCAs die verschiedensten Ursachen haben und alle am Produktionsprozess Beteiligten damit konfrontiert sind.

Der derzeitige Testlauf auf einem Prüfstand integriert das Einlaufen der Aggregate und erfasst beim kompletten Ein- und Ausfahren der Pinole während eines bestimmten Testzyklus im Bewegungsprofil Trapez Stromstärke und verschiedene Temperaturwerte. Hierdurch können Fehler aber nicht zuverlässig aufgedeckt werden, zumal keine Standards für die funktionsspezifischen Eigenschaften der NCA vorhanden sind und es kein Dokumentationssystem zur Erfassung von Problemen und deren Rückmeldung an die entsprechenden Abteilungen gibt.


Nach einer Auseinandersetzung mit den Grundlagen von Prüfkonzepten und Testverfahren werden deshalb teilweise auf der Grundlage des bisherigen Testablaufs entwickelte Testverfahren geprüft und bewertet. Dabei darf bei allen Testkonzepten der ökonomische Aspekt nicht außer Acht gelassen werden.

Das bisher schon eingesetzte langsame Abfahren im Trapezprofil ist wenig geeignet, da die Einflüsse des Reglers nicht erkennen lassen, ob eine Schwergängigkeit vorliegt. 

Aufgrund der Versuche kann davon ausgegangen werden, dass man weitreichende Erkenntnisse zu etwaigen Störungen durch das Belasten der NCAs beim Betreiben erhält. Da sich Tests mit einem direkten Belasten der Achse durch Anhängen einer Last oder Fahren auf einen Widerstand nicht als praxistauglich erweisen, werden Messungen analysiert, die aufgrund des Fahrprofils die Achsen dynamisch belasten. 

So wird das Fahrprofil Polynom 5. Ordnung neu entwickelt und berechnet. 19 NCAs werden mit dem Fahrprofil Polynom 5. Ordnung und mit dem Fahrprofil Stufenprofil gemessen und daraus Mittelwertkuren und Standardabweichungen berechnet. Bei einer Achse wird hierbei axiales Spiel hoch signifikant erkannt. Es bietet sich an, die Verfahrprofile Stufenprofil und Polynom 5. Ordnung einzusetzen, denn mit ihnen können die Achsen bis an die Belastungsgrenzen belastet werden und es lässt sich axiales Spiel erkennen. Weiterhin lassen sich diese Messungen leicht in den vorhandenen Prüfablauf integrieren.



Voraussetzung ist, dass auf eine für das Steuern der Achsen in Tests geeignete Steuerung zurückgegriffen werden kann, die auch das Aufzeichnen und Analysieren von Daten unterstützt. Dies ist mit der derzeitigen, nicht für den Testbetrieb ausgelegten VC 1 Steuerung nur sehr eingeschränkt möglich.

Als weitere Testmöglichkeit werden Schwingungsmessungen untersucht. Hierbei werden zuerst die zu erwartenden Schwingungsfrequenzen berechnet. Anschließend werden Versuche durchgeführt, um die grundsätzliche Eignung von Schwingungsmessungen als geeignete Messmethode zu evaluieren. Eine Analyse der Versuchsergebnisse zeigt, dass Schwingungsmessungen dazu geeignet sind, um Funktionsstörungen an den NCAs zu erkennen. Jedoch müssen weiterführende Versuche durchgeführt werden, um standardisierte Tests zu entwickeln.

Mit der Erfassung von Temperaturwerten wie Motortemperatur oder der Temperatur an den NCAs mithilfe externer Sensoren könnten zwar aussagefähige Ergebnisse gewonnen werden. Jedoch ist die Entwicklung praxistauglicher Messverfahren sehr aufwendig und bringt im Vergleich zu den anderen als geeignet erscheinenden Messverfahren keine entscheidenden Vorteile.


%Die Messung von Temperaturwerten wie Motortemperatur oder die Temperaturmessung mithilfe von externen Sensoren an den NCA ist im Vergleich zu den anderen untersuchten Messmethoden schwierig in der Umsetzung zu aussagekräftigen Messverfahren.





%Zudem haben Volumenstromessungen am Kühlmittel ergeben, dass man auch diese Messungen zur Analyse des Kühlsystems heranziehen kann.

Zur Analyse des Kühlmitteldurchflusses innerhalb des Aggregats wird ein Kennwert entwickelt. Aus der  Messung mehrerer Aggregate wird wiederum ein Mittelwert und dessen Standardabweichung abgeleitet,  die für das Testen zur Verfügung stehen.

Die Ergebnisse der Arbeit schaffen die Grundlage, an Hand derer das Prüfverfahren überarbeitet werden kann.

%Die Ergebnisse der Arbeit schaffen die Grundlage, an Hand derer das Prüfverfahren überarbeitet werden kann.












\begin{comment}
Die von der Firma Bihler entwickelten und produzierten NC Aggregate werden in ihren Stanz-Biegeautomaten zum Ausführen von schnellen und exakten Werzeugbewegungen eingesetzt. Da das verwendete Prüfverfahren der NCAs nicht zuverlässig Fehler bei deren Produktion in der Serie aufdeckt, wird das Verfahren analysiert und seine Schwachpunkte im Einzelnen aufgezeigt. 

Hierzu wird der Aufbau der verschiedenen Varianten der NCAs dargestellt, die elektronische Ansteuerung unter Zuhilfenahme des Kurvengetriebes erläutert und die Fahrbewegung der NCAs im Trapez- und Dreiecksprofil aufgezeigt.

Es wird festgestellt, dass Funktionsstörungen der NCAs die verschiedensten Ursachen haben und alle am Produktionsprozess Beteiligten damit konfrontiert sind.

Der derzeitige Testlauf auf einem Prüfstand integriert das Einlaufen der Aggregate und erfasst beim kompletten Ein- und Ausfahren der Pinole während eines bestimmten Testzyklus im Bewegungsprofil Trapez Stromstärke und verschiedene Temperaturwerte. Hierdurch können Fehler aber nicht zuverlässig aufgedeckt werden, zumal keine Standards für die funktionsspezifischen Eigenschaften der NCA vorhanden sind und es kein Dokumentationssystem zur Erfassung von Problemen und deren Rückmeldung an die entsprechenden Abteilungen gibt.


Nach einer Auseinandersetzung mit den Grundlagen von Prüfkonzepten und Testverfahren werden deshalb teilweise auf der Grundlage des bisherigen Testablaufs entwickelte Testverfahren geprüft und bewertet.

Dabei ist das bisher schon eingesetzte langsame Abfahren im Trapezprofil wenig geeignet, da die Einflüsse des Reglers nicht erkennen lassen, ob eine Schwergängigkeit vorliegt. 

Aufgrund der Versuche kann davon ausgegangen werden, dass man weitreichende Erkenntnisse zu etwaigen Störungen durch das Belasten der NCAs beim Betreiben erhält. Da sich Tests mit einem direkten Belasten der Achse durch Anhängen einer Last oder Fahren gegen einen Widerstand nicht als praxistauglich erweisen, werden Messungen analysiert, die aufgrund des Fahrprofils die Achsen dynamisch belasten. Es bietet sich an, die Verfahrprofile Stufenprofil und Polynom 5. Ordnung einzusetzen, denn mit ihnen können die Achsen bis an die Belastungsgrenzen belastet werden und es lässt sich axiales Spiel erkennen. Weiterhin lassen sich diese Messungen leicht in den vorhandenen Prüfablauf integrieren.

Voraussetzung ist, dass auf eine für das Steuern der Achsen in Tests geeignete Steuerung zurückgegriffen werden kann, die auch das Aufzeichnen und Analysieren von Daten unterstützt. Dies ist mit der derzeitigen, nicht für den Testbetrieb ausgelegten VC 1 Steuerung nur sehr sehr eingeschränkt möglich.

Weiterhin lassen Versuche Schwingungsmessungen als geeignet erscheinen, um Funktionsstörungen an den NCAs zu erkennen. Jedoch müssen weiterführende Versuche durchgeführt werden, um standardisierte Tests zu entwickeln.

Es zeigt sich, dass die Messung der Motortemperatur am ehesten Rückschlüsse auf das Kühlsystem zulässt, nicht jedoch auf Probleme mit dem Motor. Zudem haben Volumenstromessungen am Kühlmittel ergeben, dass man auch diese Messungen zur Analyse des Kühlsystems heranziehen kann.

Die Ergebnisse der Arbeit schaffen die Grundlage, an Hand derer das Prüfverfahren überarbeitet werden kann.
\end{comment}



\begin{comment}
Die von der Firma Bihler entwickelten und produzierten NC Aggregate werden in ihren Stanz-Biegeautomaten zum Ausführen von schnellen und exakten Werzeugbewegungen eingesetzt. Da das verwendete Prüfverfahren der NCAs nicht zuverlässig Fehler bei deren Produktion in der Serie aufdeckt, wird das Prüfverfahren analysiert und es werden Testverfahren entwickelt, geprüft und bewertet. Die Ergebnisse schaffen die Grundlage, an Hand derer das Prüfverfahren überarbeitet werden kann.


Analyse genauer und ausführlicher (mehr als üblich)

Über die als Verfahrprofile Stufenprofil und Polynom 5. Ordnung hinaus, die eine Belastung simulieren, hat sich die Schwingungsmessung als ein aussagekräftiges Testverfahren herausgestellt. 

Verfahrbewegungen
Schwingungsmessungen    

Prüfverfahren und ncas besser verkünüpft (2. Satz)

und möglichkeiten zur Erfassung der Prüfdaten aufgezeigt


Die von der Firma Bihler entwickelten und produzierten NC Aggregate werden in ihren Stanz-Biegeautomaten zum Ausführen von schnellen und exakten Werzeugbewegungen eingesetzt. Da das verwendete Prüfverfahren der NCAs nicht zuverlässig Fehler bei deren Produktion in der Serie aufdeckt, wird das das Verfahren analysiert und seine Schwachpunkte im Einzelnen aufgezeigt. Die teilweise auf der Grundlage des bisherigen Testablaufs entwickelten Testverfahren werden geprüft und bewertet. Um als geeignet erscheinende Schwingungsmessungen einsetzen zu können, müssen weiterführende Versuche durchgeführt werden. Und um die  Verfahrprofile Stufenprofil und Polynom 5. Ordnung einzusetzen, die eine Belastung der Achsen simulieren und die sich als aussagekräftig und leicht in den Ablauf integrierbar erwiesen haben, müsste eine für das Steuern der Achsen in Tests geeignete Steuerung benutzt werden, die auch das Aufzeichnen und Analysieren der Daten vereinfacht. Die Ergebnisse der Arbeit schaffen die Grundlage, an Hand derer das Prüfverfahren überarbeitet werden kann.




(ursprungsversion / Abstract)
Die von der Firma Bihler entwickelten und produzierten NC Aggregate werden in ihren Stanz-Biegeautomaten zum Ausführen von schnellen und exakten Werzeugbewegungen eingesetzt. Da das verwendete Prüfverfahren der NCAs nicht zuverlässig Fehler bei deren Produktion in der Serie aufdeckt, wird das Verfahren analysiert und seine Schwachpunkte im Einzelnen aufgezeigt. Die teilweise auf der Grundlage des bisherigen Testablaufs entwickelten Testverfahren werden geprüft und bewertet. Um die  Verfahrprofile Stufenprofil und Polynom 5. Ordnung einzusetzen, die eine Belastung der Achsen simulieren und die sich als aussagekräftig und leicht in den Ablauf integrierbar erwiesen haben, müsste eine für das Steuern der Achsen in Tests geeignete Steuerung benutzt werden, die auch das Aufzeichnen und Analysieren der Daten vereinfacht. Und um die aufgrund der Versuche als geeignet erscheinenden Schwingungsmessungen einsetzen zu können, müssen weiterführende Versuche durchgeführt werden. Die Ergebnisse der Arbeit schaffen die Grundlage, an Hand derer das Prüfverfahren überarbeitet werden kann.






\end{comment}















\begin{comment}
Die teilweise auf der Grundlage des bisherigen Testablaufs entwickelten Testverfahren werden geprüft und bewertet.

Um die  Verfahrprofile Stufenprofil und Polynom 5. Ordnung einzusetzen, die eine Belastung der Achsen simulieren und die sich als aussagekräftig und leicht in den Ablauf integrierbar erwiesen haben, müsste eine für das Steuern der Achsen in Tests geeignete Steuerung benutzt werden, die auch das Aufzeichnen und Analysieren der Daten vereinfacht. Und um die aufgrund der Versuche als geeignet erscheinenden Schwingungsmessungen einsetzen zu können, müssen weiterführende Versuche durchgeführt werden. Die Ergebnisse der Arbeit schaffen die Grundlage, an Hand derer das Prüfverfahren überarbeitet werden kann.


\end{comment}



