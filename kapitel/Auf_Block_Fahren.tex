\chapter{Auf Block Fahren}

Entscheidung fahren auf statische Last.

Pro:
\begin{itemize}
 \item Statische Haltekräfte können nicht überprüft werden
 \item Entspricht eher einem realen Belastungsfall im Einsatz
 \item Absolute Positioniergenauigkeit und Wiederholgenauigkeit kann unter Belastung erfasst werden
\end{itemize}


Contra:
\begin{itemize}
 \item (Aufwendig)
 \item Messung kann nicht einfach an anderen Maschinen wiederholt werden
  \item Arbeitssicherheit beim testen
  \item Belastungen können auch dynamisch abgebildet werden. Zur maximalen Auslastung des Aggregats ist das Fahren auf eine Last nicht erforderlich.
  \item Durch Schleppfehler kann das Spiel der Aggregate auch ohne auf Belastung zu fahren gemessen werden
  \item Die absolute Positionierung des Aggregates (Positioniergenauigkeit) ist wesentlich stärker davon abhängig ob auf Last gefahren wird oder nicht als die Wiederholgenauigkeit. Für den produktiven Einsatz ist die Wiederholgenauigkeit sehr viel wichtiger als die absolute Positioniergenauigkeit. 
  \item Verstell Mechanismus für verschiedene Hublängen nötig.
\end{itemize}

Fazit:
Der zusätzliche Gewinn durch das Fahren auf Last wird durch die Gegenargumenten nicht gerechtfertigt.
