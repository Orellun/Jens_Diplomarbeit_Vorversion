\chapter{Anforderungen an die Flugplattform}\label{cha:Anforderungen an die Flugplattform}

Um eine Zielgerichtete Entwiclung des Elektronik konzets zu ermnöglichen wurden zunächst die Globalen Anforderungen an die Flugplattform Abstrakt Formuliert. Diese wurden unter einwirkungn des sich verändernden Reglements des AUVSI Wettberwerbs und die Formulierung neuer Aufgaben für die Flugplattform über die Einsatzdauer weiterentwickelt und ergänzt.  


\section{Anforderung an das Flugzeug}

Um einen sicheren Betrieb durch sowohl den Autopiloten als auch durch einen Menschelichen Tespiloten zur ermöglichen
wurde die Fluggeschwindigkeit zu beginn zunächst auf maximal 18 m/s festgelegt. Die Geschwindikeit am Punkt des Strömungsabsrisses wird REchnerisch zu 10 m/s bestimmt.
Aus den Beschränkung beim Transport zum Wettberwerb als reguläres Gepäck im Internationalen Flugverkehr ergibt sich eine Maximale Länge aller Komponenten von 700 mm. Diese resultiert aus der größten Länge der zur verfügung stehenden Aluminium
TRansportkisten von etwa 725mm.
Ausversicherungsrechtlichen Gründen wurde das Maximale Strasrtgewicht der Plattform auf 5 Kg beschränkt.

In dieser Arbeit soll keine Detailierte betrachtung der Aerodynamischen und Mechanischen Entwirklungsresultate für das Eigentliche Flugzeug stattinden. Es wird festgehalten das alle bisher Eingesetzen Flugzeuge die oben definierten Anforderungen erfüllen und Spannweiten zwischen 1,5 m und 2,8 m Aufweisen. Des weiteren wurden Abfluggewichtw von 4 bis 4,9 Kg eingesetzt.


\section{Die Flugplattform als Sensorträger}

Um die zahlreichen Aufgaben des Wettbewerbs \begin{comment} Verweis auf Wettbewerbsaufben \end{comment}
erfüllen zu können muss die Flugplattform mit Sensoren und Verabreitungsseystemen ausgestatten werden.
Zu diesen zählen hauptsächlich das Kamerasystem zur detektierung der Buchstaben in der Suchaufgabe mit der dazugehören
Gimbal Vorrichtung zur Bildstabilisierung. Die Bilddaten sollen an Bord mit einem Kleinrechner "Odroif U4W" verarbeitet werden. Weitere Systeme sind der LIDAR Abstandssensor für einen akkuraten Landeanflug, sowie die Abwurschvorrichtung für das Ei beziehungsweise im späteren REgelwerk die Wasserflasche.
Es werden außerdem eine Reihe von Sensoren zur versorgung des Autoiloten mit Flugdaten mitgeführt. Zu diesen zählen der GPS EMpfänger, der Staudrucksensor sowie Batterie Strom- und Spannungssensoren.
Als verbindung zur bodenstation werden zwei Funkfequenzen eingesetzt.
Eine 5 Ghz Wlan Verbindung welche eine Hohe Datenrate zur übertragung von Bildern  ermöglicht jedoch aufgrund der Frequenzchakrteristik am Bioden als gefgenstelle eine Nachgeführrte Richtantenne Erfordert.
Außerdem eine Dipolantenne im Frequenzbereich 433 Mhz welche zur übertragung der Flugdaten zwischen Autopilot und Bodenstation dient. Sie ermöglicht mit der Kleineren Frequznez eine größere Verbindungereichweite bei gleicher Sendeleistung auf kosten den Datenrate.

Die Bildfrequenz der 2015 eingesetzten Kamera, Limiterite zunächst die Fluggeschwindigkeit im Reiseflug. Um eine für die weitere Auswerung der Bilder sinnvolle überlappung von etwa 20 Prozent zu erziehlen wurde die Geschwindigkeit auf 15 m/s gesetzt.

Die gesteigerte Bildfrequzenz und Auflösung des neusten Kamerasystems ermöglicht den Einsatz eine gesteigerten Missionsgeschwindikeit von 17 m/s welche für die Kommende Anwednung in der Saison 2018 geplant ist.

\section{Aufgaben der Elektronik}

Die Aufgaben der elektronik lassen sich in den Bereich der Energieverwaltrung und Verlteilung  und den Den Bereicht der Signalverteilung separieren.

Aus vorhergegangenen Test wurde das Flugsystem für eine Reine Batterielektrische Versorgung Konuzeptioniert.
Der Antrieb durch einen Verbrennungsmotor wurde für die Vorliegende Flugzeuggröße trotz besserem Energiegewicht als unzureichend zuverlässig, aufwendig handhabbar, sowie zu teuer eingestuft.
Des weiteren schränken bestehende Lärmschutzrichtlienien den Einsatzt auf Testflugplätzen ein und es wird davon ausgeganen das sich die unvermeidbaren Virbrationen negativ auf die zuverlässigkeit der Sensoren Auswirkt.

\subsection{Energieverteilung und Verwaltung}

Die Elektrische Energie aus dem Akkusystem wird primär für die Erzeugung des Vortriebs über den Antriebstrang Motorregleer, Büstenloser Motor, GEtriebe, Luftschraube verwendet.
In diesem Pfand soll die Verwendung Mehrerer Quellen (akkupacks) bei sicherer handhabung ermöglicht werden.
Zweitgrößter Energieverbrauchen sind die Aktorsysteme des Flugzeugs ("Servos") für alle Steuerflächen sowie Kamera Gimbal und Auswurfsysteme (Ei, Wasserflasche) verwendet welche mit einer Festen Spannung versorgt werden müssen.
Dritgrößter Verbraucher an Bord sind die Funksysteme im 5 Ghz und 433 Mhz Band welche ebenfalls einer Festspannungsversorung bedürfen.
Ebenfalls berücksichtigt werden sollen Verbraucher des Sensorsystems, wie Kamera und Bordcomputer mit einer Festen Spannung hoher Qualität.
Als kleinster Verbraucher bedarf der Autopilatoen Computer und seine Sensorsysteme eine Versorgung Hoher Genauigkeit sowie mit geriger Varianz.

Insgesamt sollen alle subsysteme im Fehlerfall einen Reibungslosen weiterbetrieb der Restklichen Systemteilnehmer gewährleisten. Insbesondere der Aufrechterhaltung der Funktion des Autopilotensystems wird Priorität eingeräumt.


\subsection{Signalvertreilung}



\section{Anforderung an die Elektronik}