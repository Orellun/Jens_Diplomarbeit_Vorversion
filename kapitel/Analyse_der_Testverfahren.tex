\chapter{Grundlagen von Prüfkonzepten für die NC-Aggregate}\label{cha:Grundlagen_von_Pruefkonzepten_fuer_die_NCAs}



In den vorhergehenden Kapiteln werden die NCAs und ihre Eigenschaften, ihre Funktionsstörungen, der bisherige Testablauf und dessen Ergebnisse dargestellt, also der Ist-Zustand dokumentiert. Dabei stellt sich heraus, dass momentan zum Prüfen der NCAs ein Prüfablauf verwendet wird, der nicht ausreicht, um das Ziel der Qualitätssicherung hinlänglich zu gewährleisten. Dies ist in Kapitel~\ref{ch:Kritik_Testlauf} näher dargelegt.



Ein umfassendes Prüfkonzept kann als Grundlage dienen, um die Qualitätssicherung bei einem Prüfobjekt sicherzustellen. Es umfasst dann alle zum Prüfen \cite{DIN1319_1995} notwendigen Schritte. Es sollten deshalb Zeitpunkt der Prüfung, Art der Prüfung, die zu prüfenden Merkmale bzw. Prüfkriterien, die zu prüfende Stichprobe, die Häufigkeit der Prüfung, der damit betraute Personenkreis, die Prüfmittel, das Handling, die Datenerfassung, die Dokumentation, die Datenanalyse, die Bewertung der Analyse und die sich daraus ergebenden notwendigen Konsequenzen in das Konzept einbezogen sein. Ein so weitreichendes Prüfkonzept existiert derzeit für die NCAs nicht.

\section{Grundsätzliche Aspekte von Prüfkonzepten}

Um eine Prüfplanung~\cite{Bernards2005} durchzuführen, d.h. ein umfassendes Prüfkonzept zu erstellen, wie es auch zum Prüfen der NCAs wünschenswert wäre, müssen unterschiedliche Aspekte berücksichtigt werden. So besteht das Prüfen meist aus mehreren einzelnen Tests, bei denen Testaufbau und Testablauf zusammenpassen und aufeinander abgestimmt sind. Um geeignete Prüfverfahren zu finden, sind Messgrößen zu analysieren, Prüfkriterien festzulegen und daraus abgeleitete Tests zu erstellen und zu erproben. Es sollten nicht nur quantitative Tests, bei denen objektiv durch Messen getestet wird, sondern auch qualitative Tests, bei denen subjektiv mit den Sinneswahrnehmungen getestet wird, in die Überlegungen zum Erstellen eines Prüfkonzepts einbezogen werden.

Die mit der Produktion betrauten Mitarbeiter können mit ihren Sinnen wie Sehen, Hören, Riechen oder Tasten Unregelmäßigkeiten im Herstellungsprozess entdecken, was als qualitativer Test in das Prüfkonzept mit eingebunden werden kann. Entsprechende Beobachtungen können erfasst werden, um Fehler zu reduzieren. Dafür muss die Unternehmenskultur so gestaltet sein, dass das Finden und nachhaltige Abstellen von Fehlern Wertschätzung bei den Mitarbeitern bewirkt.


Der Dokumentation sollte innerhalb eines Prüfkonzepts eine wichtige Rolle eingeräumt werden. Sie kann an verschiedenen Stellen ansetzen wie den Prüfgegenständen, dem jeweils betrauten Personal, den Umgebungsbedingungen, den Prüfungen selber, den Messergebnissen und auch der Analyse der Daten. So kann sich die Erstellung einer Fehlerdatenbank, die auch als Fehler-Wissens-Datenbank~\cite{Riebschlaeger2006} bezeichnet werden kann, als hilfreich erweisen. Damit kann erfasst werden, welche Probleme und Fehler wo aufgetreten sind und welche Maßnahmen ergriffen wurden. Damit ist es fundiert möglich, sich mit der Fehlerquelle auseinanderzusetzen. Eine spezielle Software in Form eines CAQ-Systems (Computer Aided Quality Management System) könnte bei der Anlage einer Fehlerdatenbank als probates Hilfsmittel dienen. Eine Dokumentation kann andererseits auch dazu dienen, die Prüfgegenstände qualifizierter einzusetzen. Falls Tests ergeben, dass z.B. ein NC Aggregat höher belastet werden kann, könnte es mit besseren Leistungsdaten an den Kunden weitergegeben werden. 


Wenn die ermittelten und dokumentierten Daten allen relevanten Personenkreisen zur Verfügung stehen, ist eine Verbesserung und Weiterentwicklung des Produkts auf einer fundierten Datenbasis möglich. Dies kann insbesondere helfen, die Arbeit der Konstruktionsabteilung effektiver zu gestalten. Dazu ist es im Bereich der Dokumentation unumgänglich, eine durchgehende eindeutige Kennzeichnung der hergestellten Produkte zu gewährleisten. Wenn dies von der Herstellung über die Tests, die Behebung von Fehlern bis zur Auslieferung an Kunden sichergestellt ist, können die erhobenen, festgehaltenen und ausgewerteten Daten dem Einzelprodukt eindeutig zugeordnet werden. Sollte es z.B. zu Problemen unter den Fertigungsbedingungen beim Kunden kommen, könnte dies der Konstruktionsabteilung wichtige Hinweise zur Verbesserung im Bereich der Konstruktion aber auch der gesamten Herstellung geben, wenn es diese stringente Kennzeichnung innerhalb der Dokumentation gibt.


Damit sichergestellt ist, dass ein Prüfkonzept längerfristig zur Qualitätssicherung beiträgt, sollte es flexibel konzipiert sein, um bei Änderungen im technischen Bereich angepasst zu werden und es sollte eine Evaluierung eingebaut sein, um es regelmäßig zu überprüfen und gegebenenfalls zu überarbeiten. So können sich Tests als nicht aussagekräftig herausstellen oder es werden geeignetere Tests entwickelt, die bisherige Tests ersetzen sollten.


Weiterhin sind bei der Entwicklung von Prüfkonzepten die einschlägigen gesetzlichen Vorgaben und Sicherheitsvorschriften einzuhalten. 

%So kann sich ein auf den ersten Blick geeigneter Prüfablauf als zu aufwendig oder zu kostenintensiv erweisen und kann deswegen nicht umgesetzt werden.

Da jedes Prüfen mit Kosten verbunden ist, ist eine Kalkulation von Kosten und Nutzen wichtig. Das Prüfkonzept insgesamt, aber auch seine Bestandteile sollten so gestaltet sein, dass es ökonomisch sinnvoll ist. Deshalb mag ein günstiger, nicht optimaler Test eher angeraten sein als ein Test, der zwar alle Fehler entdeckt, aber mit immensen Kosten verbunden ist. Jedoch sind rein ökonomische Überlegungen ungeeignet, wenn man mit dem Prüfen einen hohen Qualitätsstandard sichern will.    



Das Entwickeln eines komplexen Prüfkonzepts ist, neben der rein technischen Umsetzung, eine große Herausforderung, denn ein derartiges Konzept muss vom Betrieb finanziert, unterstützt, genehmigt, aufgebaut, durchgesetzt und durchgeführt werden. Diese Implementierung bedarf einer guten Vorbereitung, einer umfassenden Information aller Ebenen des Betriebes und einer möglichst breiten Einbeziehung der Mitarbeiter, wenn sie erfolgreich sein soll. Dies hängt auch davon ab, von wem das Konzept entwickelt wurde, z.B. ob intern oder extern, welche Ebene des Betriebes damit betraut ist und wo eine Notwendigkeit für das Einführen eines derartigen Konzeptes gesehen wird. Zur Implementierung ist eine Dokumentation des Prüfkonzepts in Form eines Prüfplans möglicher Weise hilfreich. Beim  Erstellen und Verarbeiten von Prüfdaten, die man als Grundlage für qualitätslenkende  Maßnahme ansehen kann, sollte man sich laut~\cite{Hering2013} auf das Notwendige beschränken, damit es von den Mitarbeitern akzeptiert wird.



\section{Anforderungen an die Prüfungen}\label{cha:Anforderungen_an_die_Pruefungen}

Der zentrale Bestandteil jedes Prüfkonzepts ist das Testen, das aus einer einzelnen Prüfung oder mehreren gleichen oder unterschiedlichen Prüfungen bestehen kann. So gibt es verschiedenste Tests und Methoden, um herauszufinden, ob die NCAs die notwendigen Eigenschaften (vgl. Kapitel~\ref{Eigenschaften_der_Aggregate}) aufweisen und somit die vorgeschriebenen Anforderungen erfüllen. Bei der Auswahl der Methoden muss darauf geachtet werden, dass man die erwarteten Ziele erreicht. Für die Prüfung in der Serienfertigung ist es besonders wichtig, dass die Methoden sehr effizient und zielgerichtet sind.


Für die Auswahl von Messmethoden in der Serienfertigung sind die in Tabelle~\ref{fig:Randbedingungen_bei_der_Auswahl_von_Messmethoden_in_der_Serienfertigung} aufgeführten Randbedingungen zu beachten. So sollten die Methoden automatisierbar und dokumentierbar sein, keinen großen Personaleinsatz erfordern und nicht zuletzt wirtschaftlich sein. Für die Mitarbeiter sollen die Tests einfach zu bedienen sein und die Mitarbeiter sollen eine klare, einfache Entscheidungsgrundlage haben. Zudem sollten die Messmethoden den Produktionsprozess nicht wesentlich verlängern, so dass eine kurze Durchlaufzeit anzustreben ist. Wie in allen sonstigen Bereichen ist streng darauf zu achten, dass bei allen Messmethoden die gesetzlichen Vorschriften eingehalten werden.











\begin{table}[H]
\center
\fbox{
\begin{minipage}[c]{0.8\textwidth}
\vspace{5pt}
\begin{itemize}
 \item kostengünstig
 \item automatisierbar
 \item gut dokumentierbar
 \item geringer Personaleinsatz
 \item einfache Bedienbarkeit
 \item klare, einfache Entscheidungsgrundlage
 \item kurze Durchlaufzeit
 \item Einhalten gesetzlicher Vorschriften
\end{itemize}

\par\vspace{2pt}
\end{minipage}}
\caption{Randbedingungen bei der Auswahl von Messmethoden in der Serienfertigung}
\label{fig:Randbedingungen_bei_der_Auswahl_von_Messmethoden_in_der_Serienfertigung}
\end{table}















\section{Tests in der Versuchsabteilung}\label{cha_Tests_in_der_Versuchsabteilung}

Wie in Kapitel~\ref{cha:Entdecker_von_Funktionsstoerungen} beschrieben, werden die Prototypen der Aggregate in der Versuchsabteilung während der Konstruktionsphase einem Prototypentest unterzogen. Bei diesem werden verschiedene, von der Konstruktionsabteilung vorgegebene Tests durchgeführt. Ziel dieser Tests ist es, die Tauglichkeit der Konstruktion festzustellen, damit die anschließend in der Serienfertigung in großer Stückzahl gefertigten NCAs keine Funktionsstörungen aufweisen. Es werden einzelne NCAs mit verschiedenen Testaufbauten getestet, was einen hohen Personalaufwand und vermehrte Umbauzeiten bedeutet. Deswegen sind die Tests aufwändig und zeitintensiv und nicht direkt auf die Serienfertigung anwendbar. Trotzdem können sie wichtige Hinweise geben, was alles getestet werden kann, wie es getestet werden kann und welche Tests über die Jahre entwickelt wurden, um daraus Ansätze für Tests der NCAs in der Serienfertigung abzuleiten.

In Tabelle~\ref{fig:Erfasste Parameter der NCAs in der Versuchsabteilung} sind die Parameter für die Funktionsfähigkeit der NC-Aggregate dargestellt, die die Versuchsabteilung erfasst hat. Diese Parameter werden außer 'Inbetriebnahme möglich' im derzeitigen Testverfahren nicht erfasst.



\begin{table}[h]
\center
\fbox{
\begin{minipage}[c]{0.8\textwidth}
\vspace{5pt}
\begin{itemize}
\item Inbetriebnahme möglich
\item kurzfristige und dauerhafte Spitzenkräfte
\item Leistungsdaten im Leerlauf 
\item Leistungsdaten mit Belastung
\item Bremshaltekraft 
\item Positionier- und Wiederholgenauigkeit
\item axiales und radiales Spiel der Pinole
\item Verhalten bei Dauerbelastung
\end{itemize}
\par\vspace{2pt}
\end{minipage}}
\caption{Erfasste Parameter der NCAs in der Versuchsabteilung}
\label{fig:Erfasste Parameter der NCAs in der Versuchsabteilung}
\end{table}




Tabelle~\ref{fig:Messmethoden_in_der_Versuchsabteilung} zeigt die Messmethoden auf, mit denen die in Tabelle~\ref{fig:Erfasste Parameter der NCAs in der Versuchsabteilung} dargestellten Parameter in der Versuchsabteilung erfasst werden. Dabei kann man z.T. mit einer Messmethode mehrere Parameter erfassen. 




\begin{table}[h]
\center
\fbox{
\begin{minipage}[c]{0.8\textwidth}
\vspace{5pt}
\begin{itemize}
\item Anschließen der Achse an die Steuerung
\item Steuerbarkeit der Achse ohne Fehler
\item Fahren auf Block mit unterschiedlichen Haltezeiten
\item Abfahren des Bewegungsprofils mit Gasdruckfeder als Belastung
\item Abfahren des Bewegungsprofils im Leerlauf 
\item Aufbringen von Kräften, bis die Haltebremse die aufgebrachte Kraft nicht mehr halten kann
\item Messung der Position der Pinole mit Wirbelstromsensoren mit und ohne Belastung
\item Messen des radialen und axialen Spiels durch Aufbringen von Kräften
\end{itemize}
\par\vspace{2pt}
\end{minipage}}
\caption{Messmethoden in der Versuchsabteilung}
\label{fig:Messmethoden_in_der_Versuchsabteilung}
\end{table}


%Kurze Analyse der Messmethoden aus Tabelle~\ref{fig:Messmethoden_in_der_Versuchsabteilung}. Genauere Darstellung in Kapitel~\ref{cha:Pruefkonzepte} falls Messmethode interressant Aussuchen welche Tests für die Serienprüfung interessant sind


Mit der Methode des 'Anschließens der Achse an die Steuerung' wird der Parameter 'Inbetriebnahme möglich' getestet. Es handelt sich bei diesem Test wie in Kapitel~\ref{ch:Kritik_Testlauf} erläutert, um die Grundvoraussetzung zum Betreiben der NCAs. Man kann feststellen, ob die Achse ohne Fehler betrieben werden kann. Dieser grundlegende Test ist somit nicht nur in der Ver\-suchs\-pha\-se enthalten, sondern auch im bisherigen Prüfverfahren, und ist aus keinem Prüfkonzept wegzudenken.



%wird derzeit im Serientest nicht gemacht. interessante Methode um Belastungen zu testen, Spitzenlasten und Dauertest, Leistungsdaten mit Belastung. näheres in Kapitel~\ref{cha:Fahren auf eine Feder oder auf Block}


Mit den beiden Messmethoden 'Fahren auf Block mit unterschiedlichen Haltezeiten' und dem 'Abfahren eines Bewegungsprofils mit Gasdruckfeder' handelt es sich um Belastungstests, also um Tests, bei denen die Pinole höheren Kräften ausgesetzt ist. Man kann damit sowohl die in Tabelle~\ref{fig:Erfasste Parameter der NCAs in der Versuchsabteilung} aufgeführten Parameter der 'kurzfristigen und dauerhaften Spitzenkräfte' testen genauso wie den Parameter 'Leistungsdaten mit Belastung' als auch den Parameter 'Verhalten bei Dauerbelastung'. Es ist eine interessante Messmethode, um Leistungsdaten mit Belastung zu testen. Deshalb wird sie im Kapitel~\ref{cha:Fahren auf eine Feder oder auf Block} als mögliche Testmethode aufgeführt.

%Spitzenlasten und Dauertest, Leistungsdaten mit Belastung

Die Messmethode 'Abfahren von Bewegungsprofilen im Leerlauf' wird nicht nur im Versuchsbereich eingesetzt, sondern auch in den derzeit benutzten Prüfungen gefahren, um die 'Leistungsdaten im Leerlauf' zu messen. Da sich die Messung mit geringem Aufwand durchführen lässt und hilfreiche Leistungsdaten liefert, ist zu erwägen, ob sie in einem neuen Prüfkonzept für die NCAs enthalten sein sollte (Vergleiche Kapitel~\ref{cha:Abfahren eines Profils im Leerlauf}). Darüber hinaus ist erwägenswert, ob mit dieser Messmethode durch weitere Messungen oder durch Änderungen am Fahrprofil weitreichendere Informationen zur Funktionsfähigkeit der NCAs zu gewinnen wären.





%Abfahren von Bewegungsprofilen im Leerlauf. wird derzeit schon gemacht. ist auch weiter interessant, geringer Aufwand. Vergleiche Kapitel~\ref{cha:Abfahren eines Profils im Leerlauf}, Leistungsdaten



Während der Entwicklungsphase der NCAs ist das Testen der Haltebremse und deren Bremshaltekraft wichtig. Getestet wird, indem man so starke Kräfte aufbringt, bis die Haltebremse die aufgebrachte Kraft nicht mehr halten kann. Es handelt sich um einen sehr aufwendigen Test wie in Kapitel~\ref{cha_Messung_der_Bremshaltekraft} ausgeführt.

%Da derzeit keine Probleme mit der Haltebremse bekannt sind, besteht keine Notwendigkeit, diesen aufwendigen Test für das Prüfkonzept in der Serienfertigung vorzuschlagen.


%Messung der Haltebremse, Bremshaltekraft, Aufbringen von Kräften bis die Haltebremse die aufgebrachte Kraft nicht mehr halten kann, Derzeit keine Probleme mit der Haltebremse, deswegen derzeit nicht in der Serienfertigung nötig.






In Tabelle~\ref{fig:Messmethoden_in_der_Versuchsabteilung} ist die 'Messung der Position der Pinole mit Wirbelstromsensoren mit und ohne Belastung' als Messmethode während der Entwicklungsphase der NCAs zu finden, um den Parameter 'Positionier- und Wiederholgenauigkeit' (siehe Tabelle~\ref{fig:Erfasste Parameter der NCAs in der Versuchsabteilung}) zu testen. Dabei wird das wiederholte Anfahren derselben Position mithilfe von Wirbelstromsensoren gemessen. Es handelt sich um eine aufwendige externe Messung, deren Ergebnisse mehr von den Einstellungen des Motorreglers abhängen als vom Aggregat selber. Darüber hinaus können beim Zweigebersystem Unregelmäßigkeiten direkt erkannt werden. Deshalb kommt diese Messung nicht in die engere Wahl für einen Test in der Serienprüfung.




%an Motorgeber Einstellungen abhängig als vom Aggregat selber. Aufwendige externe Messun Erfahrungsgemäß mehr von den Motorgeber Einstellungen abhängig als vom Aggregat selber. Aufwendige externe Messung. Durch zweigeber System kann auch so unregelmäßigkeiten festgestellt werden.






Mit dem 'Messen des radialen und axialen Spiels' durch Aufbringen von Kräften soll der Parameter 'axiales und radiales Spiel der Pinole' gemessen werden. Zu diesem Verfahren wurden Untersuchungen im Betrieb gemacht und ein fertiges Konzept entwickelt. Jedoch kommt das Verfahren beim derzeitigen Testen nicht zum Einsatz. Es ist sehr zeit-, personal- und kostenaufwendig und bringt keinen diesem Aufwand adäquaten Informationsgewinn. Deshalb erfüllt es die Kriterien zur Auswahl von Messmethoden in der Serienfertigung (Tabelle~\ref{fig:Randbedingungen_bei_der_Auswahl_von_Messmethoden_in_der_Serienfertigung}) nicht.




\section{Erweiterung der Messgrößen}\label{cha_Moegliche_Messgroeen}

Außer den in der Versuchsabteilung während der Konstruktionsphase der NCAs erfassten Messgrößen (vgl. Tabelle~\ref{fig:Erfasste Parameter der NCAs in der Versuchsabteilung}) und den in Kapitel~\ref{cha:Ermittelte_Messwerte} beschriebenen, derzeit im Testablauf erfassten Größen, Temperatur mit außen auf den Achsen aufgesetzten Temperatursensoren, Motortemperatur und maximale Stromstärke bei konstanter Geschwindigkeit, gibt es weitere eventuell neu zu entwickelnde Testverfahren, Messmethoden und Messgrößen, bei denen abzuwägen ist, ob sie als Bestandteil eines Prüfkonzeptes sinnvoll sind. Denn aus Kapitel~\ref{ch:Kritik_Testlauf} ist ersichtlich, dass das zur Zeit angewandte Testverfahren einer Umgestaltung bedarf.

Als Grundlagen für diese Umgestaltung sind Recherchen durchzuführen, die über das Analysieren der in der Versuchsabteilung gefahrenen Tests hinausgehen. Aus ihnen ergibt sich, dass Messgrößen aus unterschiedlichen Bereichen einbezogen werden müssen.

% Als Grundlagen für diese Umgestaltung zu schaffen, ist es notwendig, Recherchen über das Analysieren der in der Versuchsabteilung gefahrenen Tests hinaus durchzuführen. Aus ihnen ergibt sich, dass Messgrößen aus unterschiedlichen Bereichen einbezogen werden müssen.

\begin{itemize}
 \item  Insbesondere kämen als mögliche Messung Schwingungsfrequenzanalysen in Frage, denn mit Schwingungsmessungen können im Maschinenbau Rückschlüsse auf Eigenschaften bzw. Unstimmigkeiten gezogen werden.
 
 \item Ähnlich verhält es sich mit Akustikmessungen, die generell bei der Abnahme von Maschinen verwendet werden. 
 
 \item Temperaturmessungen sind eine weitere bewährte Prüfmethode, denn sie liefern Daten zum thermischen Verhalten und zum Energieumsatz, wobei eine erweiterte Termperaturmessung einen Fehler aufzeigen könnte. 
 
 % \item Als mögliche Messgröße kämen Schwingungsfrequenzen in Frage, da man im Maschinenbau allgemein damit Rückschlüsse auf Eigenschaften bzw. Unstimmigkeiten zieht. Ähnlich verhält es sich mit Schallmessungen, die generell bei der Abnahme von Maschinen verwendet werden. Temperaturmessungen sind eine weitere bewährte Prüfmethode und sie würden Daten zum thermischen Verhalten und zum Energieumsatz liefern, wobei eine Temperaturerhöhung auf einen Fehler hindeuten könnte. 
 
 \item Mit Durchflussmessungen könnte man die Funktionsfähigkeit von Kühlung und Schmierung überprüfen. Auch das Überprüfen der Dichtigkeit der Schmierung wäre ein Ansatzpunkt für einen Test, der optimaler Weise über einen längeren Zeitraum wiederholt durchgeführt werden sollte.
 
 \item Tests in Bezug auf mechanisches Spiel würden Aussagen zur mechanischen Präzision liefern. Tests zur Wiederhol- und Positioniergenauigkeit wären wichtig für Aussagen über die Prozesssicherheit des Aggregats, da sie Aufschluss über die Steifigkeit und Genauigkeit des Antriebs geben könnten.


 \item Die Dokumentation der von der VC 1 Steuerung während des Betriebes der NCAs erfassten Messgrößen wie Motorstrom und Motortemperatur könnte eine Datengrundlage für die Funktionstüchtigkeit der NCAs liefern.
\end{itemize}


















\begin{comment}

\colorbox{orange}{TEXT in Kap 6}
Von der VC 1 Steuerung werden verschiedene Werte während des Betriebes der NCAs zwar erfasst jedoch nicht angezeigt und dokumentiert wie in Kapitel~\ref{} erklärt. Sie dienen dem präzisen Betreiben der NCAs. Es handelt sich dabei um den Motorstrom, die Motortemperatur, die Fequenz des Erregerfeldes, den Positionswert des Motorgebers und falls vorhanden den Positionswert des Linearmesssystem.

So sind aus dem Motorstrom, wie in Kapitel~\ref{ch:Ermittelte_Messwerte_Stromstaerke} erklärt, indirekt Drehmoment und Kraft bei konstanter Spannung näherungsweise berechenbar. Die Motortemperatur gibt Hinweise auf die thermische Auslastung. Die Frequenz des Erregerfeldes beim Synchronmotor lässt auf die Motordrehzahl und die Winkelposition schließen. Aus dem Positionswert des Motorgebers lässt sich die rotatorische Position des Motors und die Drehzahl berechnen. Und über die Steigung ist die lineare Position der Pinole berechenbar. Mit dem Linearmesssystem beim 2-Gebersystem (vgl. Kapitel~\ref{cha:Unterschiede der NC-Aggregate}) wird die lineare Position der Pinole bestimmt.

Wenn man durch eine erweiterte Dokumentation dieser von der VC 1 erfassten Werte die so geschaffene Datengrundlage während des möglichen Prüfprozesses auswerten könnte, ließen sich daraus wichtige Kennwerte der NCAs ableiten. Diese ständen als Testwerte für Aussagen über die Funktionsfähigkeit der NCAs zur Verfügung ohne dass externe Messeinrichtungen nötig wären.


\colorbox{orange}{für Kap 6 text}

iNBETRIEBNAHME



\section{}


Aus den Vorüberlegungen zu möglichen Tests für die NCAs, ergeben sich Prüfungen, die an NCAs auf dem derzeitigen Prüfstand im Rahmen dieser Arbeit versuchsweise durchgeführt werden.
Dabei handelt es sich um:






Abfahren eines Profils unter Belastung
Fahren auf eine Feder oder auf Block
Anhängen eines Gewichtes oder einer Last
Abfahren eines Profils im Leerlauf
Langsames Abfahren im Trapezprofil
Abfahren im Stufenprofil
Abfahren im Maximalprofil Polynom 5 Ordnung
Schwingungsmessung
Temperaturmessung
Messung der Motortemperatur
Messung der Temperatur mit externen Sensoren
Volumenstrommessung
Volumenstrommessung des Kühlwassers    
Volumenstrommessung des Schmieröls
Messung der Bremshaltekraft

\end{comment}