\chapter{Fazit und Ausblick}



Sowohl die Analyse des Produktionsprozesses der NCAs als auch  die Analyse der beim Kunden erfassten Daten zeigt, dass Funktionsstörungen bei den NCAs ein relevantes Problem darstellen. Folgerichtig ergibt die Analyse des derzeitigen Testprogramms, dass zwar Fehler entdeckt werden, es aber bei weitem nicht ausreicht, eine qualitativ hochwertige Serienproduktion sicherzustellen.


Als gravierendste Mankos des derzeitigen Testverfahrens haben sich eine fehlende Erfassung, Dokumentation und Auswertung von relevanten Daten über Fehler bei den NCAs, eine nicht durchgängige Kennzeichnung der Aggregate, das Erfassen nicht aussagekräftiger Prüfkriterien beim Testverfahren und das Arbeiten mit der VC 1 Steuerung, die für diesen Zweck nur marginal geeignet ist, erwiesen.

Der derzeitige Prüfstand kann weiter verwendet werden, jedoch müssen darauf Tests gefahren werden, mit denen wesentlich aussagekräftigere Kennwerte erfasst werden.  Auch der Versuchsablauf, d.h. die Aufeinanderfolge verschiedener Testläufe, kann in wesentlichen Teilen beibehalten werden, muss aber durch weitere Tests ergänzt werden. Im Versuchsablauf  sollte weiterhin die Einlaufphase des Gewinderollentriebs mit enthalten bleiben. Der rein technische Testablauf und das Erfassen von Kennwerten reichen für ein aussagekräftiges Prüfverfahren nicht aus, denn die Qualität der Testergebnisse hängt entscheidend vom Testumfeld ab, so u.a. wie der Test\-ab\-lauf organisiert ist, wer für Messungen zuständig ist und in welcher Form diese dokumentiert werden.

%Der derzeitigen Versuchsaufbau ist so gestaltet, dass er wesentlich aussagekräftigere Kennwerte zu erfassen. 


% Zusammenfassend ist durch den derzeitigen Testablauf nicht sichergestellt, dass die NCA's die funktionsrelevanten Eigenschaften aufweisen. Deshalb muss die Teststrategie grundlegend überarbeitet werden.

% Eine Einlaufphase des Gewinderollentriebs ist unabdingbar notwendig und sie sollte mit der derzeitigen Aufteilung der Zyklen beibehalten werden. Auch wenn die Einlaufzyklen beibehalten werden, müssen die derzeit damit verbundenen Tests neu gestaltet werden.




In der Firma Bihler gibt es keine Dokumentation über Fehler, die innerhalb des Produktionsprozesses auftreten. Nur über massive, beim Kunden auftretende Störungen, die meist zu einem Totalausfall führen, gibt es Berichte, denn sie werden über ein Reklamationsmanagementsystem erfasst. Es gibt keine Strukturen, diese Daten innerhalb des Betriebes weiterzugeben und sie werden deshalb nicht für Verbesserungen genutzt. Zudem existiert kein durchlaufendes Kennzeichnungssystem für die NCAs, sodass Fehler nicht zurückverfolgt und zugeordnet werden können. Mit einer Vernummerung der Einzelteile, einer fortlaufenden Nummer für die Aggregate und hinterlegten Prüfprotokollen wären die Grundvoraussetzung für ein strukturiertes Verbessern der NCAs und des Produktionsprozesses gegeben. Ein verbesserter Testprozess kann über das direkte Auffinden von Fehlern hinaus zu einer Optimierung der NCAs und deren Produktion beitragen, wenn man auf die Dokumentation und Auswertung der erfassten Werte ein starkes Augenmerk richtet. Optimal wäre hier das Anlegen einer Fehlerdatenbank, in der die Daten der NCAs mit ihren Fehlern erfasst werden und auf die alle im Produktionsprozess Eingebundenen zugreifen können.


Der derzeitge Aufbau der VC 1 Steuerung macht einen sinnvollen Prüfablauf und eine automatische Dokumentation der Messwerte schwierig bis unmöglich, denn die Steuerung ist nicht für die Programmierung und Datenerfassung bei Testabläufen konzipiert. Eine mögliche Alternative wäre die Verwendung des reinen Automationstudios ohne den Overhead der VC 1 Steuerung. Dann kann man auch die vielen, von der Steuerung während des Fahrens generierten Daten erfassen und auswerten und hat so ohne weiteren Aufwand verschiedenste Kenndaten zu jedem getesten NCA zur Verfügung. Zudem ist das momentan verwendete Messprogramm zur Erfassung der Prüfwerte sehr schwierig zu warten, so dass schon deshalb Handlungsbedarf besteht.


% Ein großes Manko des beim zur Zeit gefahrenen Tests ist, dass es sich um keinen Belastungstest handelt. Auf dem derzeitigen Testprüfstand werden deshalb verschiedene Möglichkeiten erprobt, geeignete Belastungstests zu finden. Die aufwendigen mechanischen Prüfungen wie Fahren auf Block oder auf Feder sind ebenso wie die Tests, bei denen mechanische Umbauten notwendig sind, wie beim Test Anhängen eines Gewichtes oder einer Last weniger geeignet. \colorbox{orange}{Begründung fehlt (weil)} Als sehr aussagekräftig und wesentlich weniger risikoreich und weniger personalintensiv zeigen sich die dynamischen Tests, bei denen die Belastung steuerungstechnisch durch entsprechende Verfahrprofile aufgebracht wird. Untersucht werden hier das Abfahren eines Profils im Leerlauf, und zwar das Abfahren im Stufenprofil oder im Maximalprofil Polynom 5. Ordnung. Für die NCA 4 p = 10 sind die Messwerte schon erfasst, während für andere NCA Typen noch Messwerte gesammelt werden müssen. Weitere Versuche zeigen, dass zum Erkennen von axialem und radialem Spiel auf das aufwendige Messen von axialem und radialem Spiel verzichtet werden kann. Denn mit dem Abfahren in Stufenprofil oder im Maximalprofil Polynom 5. Ordnung kann axiales Spiel problemlos erkannt werden.





Ein großes Manko des beim zur Zeit gefahrenen Tests ist, dass es sich um keinen Belastungstest handelt. Auf dem derzeitigen Testprüfstand werden deshalb verschiedene Möglichkeiten erprobt, geeignete Belastungstests zu finden. Die mechanischen Prüfungen wie Fahren auf Block oder auf Feder sind ebenso wie die Tests Anhängen eines Gewichtes oder einer Last weniger geeignet, da sie aufwendig sind, zumal sie z.T. mechanische Umbauten erfordern. Als sehr aussagekräftig, wesentlich weniger risikoreich und weniger personalintensiv zeigen sich die dynamischen Tests, bei denen die Belastung steuerungstechnisch durch entsprechende Verfahrprofile aufgebracht wird. Untersucht werden hier das Abfahren eines Profils im Leerlauf, und zwar sowohl das Abfahren im Stufenprofil als auch im Maximalprofil Polynom 5. Ordnung. Für die NCA 4 p = 10 sind die Messwerte schon erfasst, während für andere NCA-Typen noch Messwerte gesammelt werden müssen. Bei einem der getesteten NCAs ist axiales Spiel vorhanden und dieses kann mit den beiden entwickelten, eine Belastung simulierenden Verfahrprofilen erkannt werden.  So zeigen die Versuche, dass zum Erkennen von axialem und radialem Spiel auf das aufwendige Messen von axialem und radialem Spiel verzichtet werden kann. Mit den getesten Prüfverfahren Abfahren im Stufenprofil und Abfahren im Maximalprofil Polynom 5. Ordnung, die einen echten Belastungstest darstellen, stehen Tests im Rahmen des derzeitigen Prüfaufbaus und Prüfablaufs zur Verfügung, die bei entsprechender Anpassung der VC 1 Steuerung eine zuverlässigere Testung ermöglichen.
 
 %Denn mit dem Abfahren im Stufenprofil oder im Maximalprofil Polynom 5. Ordnung kann axiales Spiel problemlos erkannt werden. 


%Mithilfe dieser Tests konnte ein NCA mit axialem Spiel erkannt werden und somit ein fehlerhaftes NCA repariert werden.



%Mit den darüber hinaus entwickelten und getesten Prüfverfahren Abfahren im Stufenprofil und Abfahren im Maximalprofil Polynom 5. Ordnung, die einen echten Belastungstest darstellen, stehen Tests im Rahmen des derzeitigen Prüfaufbaus und Prüfablaufs zur Verfügung, die bei entsprechender Anpassung der VC 1 Steuerung eine zuverlässigere Testung ermöglichen.


% In weiteren Versuchen werden die Grundlagen von Schwingungsmessungen zur Analyse der NCAs erarbeitet. 

Des Weiteren werden Grundlagen für die Schwingungsuntersuchungen zur Analyse der NCAs erarbeitet. Diese Grundlagen werden mithilfe von Tests verifiziert. Allerdings sind weitere Untersuchungen in dieser Richtung notwendig. Durch die Messung an NCAs mit gezielt eingebauten Fehlern müssen in Zukunft Richtwerte ermittelt werden. Das Einsetzen von Schwingungsmessungen erweist sich aufgrund der durchgeführten Versuche als möglich und sinnvoll. Nicht zu unterschätzen sind die Möglichkeiten, die standardisierte Tests wie die Schwingungsmessungen langfristig nicht nur innerhalb des Produktionsprozesses der NCAs sondern darüber hinaus für die Wartung und Fehleranalyse beim Kunden vor Ort eröffnen.

%Um Richtwerte zu erhalten,  man vor dem Messen gezielt Fehler in die NCAs einbauen.


%\colorbox{orange}{Umschreiben keine möglichkeitsform}Das Einsetzen von Schwingungsmessungen erscheint aufgrund der durchgeführten Versuche als möglich und sinnvoll. Allerdings sind weitere Untersuchungen in dieser Richtung notwendig. Um Richtwerte zu erhalten, könnte man vor dem Messen gezielt Fehler in die NCAs einbauen.

Zur Untersuchung des Kühlkreislaufes innerhalb der NCAs bietet sich als Test eine Volumenstrommesung des Kühlwassers an. Sie ist eine leicht automatisierbare, sinnvolle Messung, die in verbesserten Tests zum Einsatz kommen sollte. Als Referenzwert kann die in den dazu durchgeführten Versuchen ermittelte Kennzahl verwendet werden.


%Eine weitere, leicht automatisierbare, sinnvolle Messung, die in verbesserten Tests zum Einsatz kommen sollte, ist die Volumenstrommessung des Kühlwassers. Dazu kann die in Versuchen errechnete Kennzahl verwendet werden.






Innerhalb der Arbeit können Testmöglichkeiten für die Serienfertigung der NCAs zwar analysiert und entwickelt werden, die das derzeitige Prüfverfahren zuverlässiger machen. Jedoch sind weitere Analysen und Messungen nötig, um daraus Prüfkriterien abzuleiten, an Hand derer entschieden werden kann, ob ein Aggregat den Anforderungen entspricht. Wünschenswert wäre darüber hinaus ein komplexes Prüfkonzept zu entwickeln, das sämtliche Bereiche des Prüfens einbezieht und die Nutzung der erfassten Daten umfassend für die Qualitätssicherung und Weiterentwicklung der NCAs zur Verfügung stellt. 


Die Arbeit zeigt insgesamt, dass mit dem derzeitigen Prüfverfahren selbst gravierende Fehler bei der Serienproduktion der NC-Aggregate nicht zuverlässig entdeckt werden. Somit ist das Prüfverfahren dringend überarbeitungsbedürftig. Die im Rahmen der Arbeit entwickelten Prüfverfahren stellen eine geeignete Grundlage zum Erstellen eines umfassenden Prüfkonzepts dar.

























\begin{comment}

Die Firma Bihler entwickelt und produziert für ihre Stanz-Biegeautomaten NC-Aggregate. Mit diesen werden Werkzeuge präzise bedient, um Bauteile und Baugruppen in der Massenproduktion zu fertigen. Um zu vermeiden, dass es zu Funktionsfehlern bei den in Serie gefertigten NCAs kommt, wird durch Prüfen während des Produktionsprozesses versucht, derartige Probleme so weit wie möglich zu vermeiden. Obwohl die Aggregate derzeit ein Prüfverfahren durchlaufen, werden nicht alle gravierenden Fehler entdeckt. Deshalb soll das derzeitige Prüfen analysiert und Vorschläge zu einer eventuell nötigen Überarbeitung entwickelt werden.


Als Grundlage dazu dient die Auseinandersetzung mit der Konstruktion, der Produktion und dem Material der NCAs und so werden zunächst der Aufbau, die Funktionsweise und die Eigenschaften der NC-Aggregate beschrieben. Es handelt sich bei den NCAs um technisch komplexe Baugruppen, so dass in den verschiedensten Bereichen Fehler auftreten können. Derartige mögliche Fehler müssen erfasst werden, um einen Überblick zu bekommen, welche Fehler, wann, wo, wie häufig auftreten und wie gravierend diese sind. Unter dem ökonomischen Aspekt muss man die möglichen Kosten, die ein Fehler verursacht, dem gegenüberstellen, welche Kosten für das Entdecken und Beseitigen dieses Fehlers entstehen.

Im Allgemeinen können erst nach Erfassen dieser Grundlagen daraus Tests abgeleitet werden, die dazu beitragen, Störungen frühzeitig zu erkennen und zu beheben, so dass das Produkt NC Aggregat den Qualitätsansprüchen der Firma gerecht wird und sich die Wirtschaftlichkeit der Produktion erhöht. 


Jedoch gibt es in der Firma Bihler keine Dokumentation über Fehler, die innerhalb des Produktionsprozesses auftreten. Nur über massive, beim Kunden auftretende Störungen, die meist zu einem Totalausfall führen, gibt es Berichte, denn sie werden über ein Reklamationsmanagementsystem erfasst. Es gibt keine Strukturen, diese Daten innerhalb des Betriebes weiterzugeben und sie werden deshalb nicht für Verbesserungen genutzt. 

Zudem existiert kein durchlaufendes Kennzeichnungssystem für die NCAs, sodass Fehler nicht zurückverfolgt und zugeordnet werden können. Mit einer Vernummerung der Einzelteile, einer fortlaufenden Nummer für die Aggregate und hinterlegten Prüfprotokollen wären die Grundvoraussetzung für ein strukturiertes Verbessern der NCAs und des Produktionsprozesses gegeben.

Nachdem die Analyse der beim Kunden erfassten Daten zeigt, dass Funktionsstörungen bei den NCAs ein relevantes Problem darstellen, ergibt folgerichtig die Analyse des derzeitigen Testprogramms, dass es zwar gute Ansätze bietet und Fehler entdeckt werden, es aber weitreichendes Verbesserungspotential aufweist. So wäre es möglich, mit dem derzeitigen Versuchsaufbau wesentlich aussagekräftigere Kennwerte zu erfassen. Auch der Versuchsablauf, der optimaler Weise zugleich die Einlaufphase des Gewinderollentriebs einschließt, könnte in wesentlichen Teilen beibehalten und wo nötig ergänzt werden.

Ein großes Manko des beim zur Zeit gefahrenen Tests ist, dass es sich um keinen Belastungstest handelt. Auf dem derzeitigen Testprüfstand werden deshalb verschiedene Möglichkeiten erprobt, geeignete Belastungstests zu finden. Die aufwendigen mechanischen Prüfungen wie Fahren auf Block oder auf Feder sind ebenso wie die Tests, bei denen mechanische Umbauten notwendig sind, wie beim Test Anhängen eines Gewichtes oder einer Last weniger geeignet. \colorbox{orange}{Begründung fehlt (weil)} Als sehr aussagekräftig und wesentlich weniger risikoreich und weniger personalintensiv zeigen sich die dynamischen Tests, bei denen die Belastung steuerungstechnisch durch entsprechende Verfahrprofile aufgebracht wird. Untersucht werden hier das Abfahren eines Profils im Leerlauf, und zwar das Abfahren im Stufenprofil oder im Maximalprofil Polynom 5. Ordnung. Für die NCA 4 p = 10 sind die Messwerte schon erfasst, während für andere NCA Typen noch Messwerte gesammelt werden müssten.

Weitere Versuche zeigen, dass zum Erkennen von axialem und radialem Spiel auf das aufwendige Messen von axialem und radialem Spiel verzichtet werden kann. Denn mit dem Abfahren in Stufenprofil oder im Maximalprofil Polynom 5. Ordnung kann axiales Spiel problemlos erkannt werden.


Das Einsetzen von Schwingungsmessungen erscheint aufgrund der durchgeführten Versuche als möglich und sinnvoll. Allerdings sind weitere Untersuchungen in dieser Richtung notwendig. Um Richtwerte zu erhalten, könnte man vor dem Messen gezielt Fehler in die NCAs einbauen.


Eine weitere, leicht automatisierbare, sinnvolle Messung, die in verbesserten Tests zum Einsatz kommen sollte, ist die Volumenstrommessung des Kühlwassers. Dazu kann die in Versuchen errechnete Kennzahl verwendet werden.

Ein verbesserter Testprozess könnte über das direkte Auffinden von Fehlern hinaus zu einer Optimierung der NCAs und deren Produktion beitragen, wenn man auf die Dokumentation und Auswertung der erfassten Werte ein starkes Augenmerk richten würde. Optimal wäre hier das Anlegen einer Fehlerdatenbank, in der die Daten der NCAs mit ihren Fehlern erfasst würden und auf die alle im Produktionsprozess Eingebundenen zugreifen könnten.

Jedoch macht der derzeitge Aufbau der VC 1 Steuerung einen sinnvollen Prüfablauf und eine automatische Dokumentation der Messwerte schwierig bis unmöglich, denn die Steuerung ist nicht für die Programmierung und Datenerfassen bei Testabläufen konzipiert. Eine mögliche Alternative wäre die Verwendung des reinen Automationstudios ohne das Overheads der VC 1 Steuerung. Dann könnte man auch die vielen, von der Steuerung während des Fahrens generierten Daten erfassen und auswerten und hätte so ohne weiteren Aufwand verschiedenste Kenndaten zu jedem getesten NCA zur Verfügung. Zudem ist das momentan verwendete Messprogramm zur Erfassung der Prüfwerte fast nicht wartbar, so dass schon deshalb Handlungsbedarf besteht.


Die Erfahrungen im derzeitigen Testablauf haben gezeigt,\colorbox{orange}{was wo wie gezeigt, spezifischer} dass der rein technische Testablauf und das Erfassen von Kennwerten für ein aussagekräftiges Prüfverfahren nicht ausreichen. Die Qualität der Testergebnisse hängt entscheidend mit vom Testumfeld ab, so u.a. wie der Test\-ab\-lauf organisiert ist, wer für Messungen zuständig ist und in welcher Form diese dokumentiert werden.


Im Rahmen dieser Arbeit konnten Probleme im Zusammenhang mit dem Prüfprozess der NC-Aggregate in der Serienfertigung aufgezeigt werden. Als gravierende Mankos haben sich eine fehlende Erfassung, Dokumentation und Auswertung von relevanten Daten über Fehler bei den NCA, eine nicht durchgängige Kennzeichnung der Aggregagte, das Erfassen nicht aussagekräftiger Prüfkriterien beim Testverfahren und das Arbeiten mit der VC 1 Steuerung, die für diesen Zweck nur marginal geeignet ist, erwiesen.

Mit den darüber hinaus entwickelten und getesten Prüfverfahren Abfahren im Stufenprofil und Abfahren im Maximalprofil Polynom 5. Ordnung, die einen echten Belastungstest darstellen, stehen Tests im Rahmen des derzeitigen Prüfaufbaus und Prüfablaufs zur Verfügung, die bei entsprechender Anpassung der VC 1 Steuerung eine zuverlässigere Testung ermöglichen würden.


Innerhalb der Arbeit konnten Testmöglichkeiten für die Serienfertigung der NCAs zwar analysiert und entwickelt werden, die das derzeitige Prüfverfahren zuverlässiger machen würden. Jedoch sind weitere Analysen und Messungen nötig, um daraus Prüfkriterien abzuleiten, an Hand derer entschieden werden kann, ob ein Aggregat den Anforderungen entspricht. Wünschenswert wäre darüber hinaus ein komplexes Prüfkonzept zu entwickeln, das sämtliche Bereiche des Prüfens einbezieht und die Nutzung der erfassten Daten umfassend für die Qualitätssicherung und Weiterentwicklung der NCAs zur Verfügung stellt. 


Nicht unterschätzen sollte man die Möglichkeiten, die standardisierte Tests wie die Schwingungsmessungen langfristig nicht nur innerhalb des Produktionsprozesses der NCAs sondern darüber hinaus für die Wartung und Fehleranalyse beim Kunden vor Ort eröffnen könnten.


Insgesamt darf aber bei allen Testkonzepten der ökonomische Aspekt nicht außer acht gelassen werden. \colorbox{orange}{anderer Schlusssatz}

\end{comment}
