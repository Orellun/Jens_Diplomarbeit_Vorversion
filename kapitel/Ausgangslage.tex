\chapter{Einleitung}


\section{Motivation}

Die Otto Bihler Maschinenfabrik GmbH \& Co. KG ist ein Systemlieferant in der Stanzbiege"~, Schweiß- und Montagetechnik. Seit 60 Jahren setzt sich die Firma mit zukunftsweisenden Fertigungssystemen und -lösungen auseinander. Es handelt sich um  ein mittelständisches Unternehmen mit weltweit etwa 900 Beschäftigten. Außer am Hauptsitz in Halblech im Allgäu wird an zwei weiteren Standorten, in Füssen im Allgäu und in den USA (Phillipsburg, New Jersey), produziert. 

Die 1953 von dem Flugzeugmechaniker Otto Bihler eröffnete Werkstatt zur Fertigung von Federn mündete über die Entwicklung der ersten radialen Draht- und Bandbiegeautomaten 1958 in die Gründung der Otto Bihler Maschinenfabrik KG. 

Bei der Stanz-Biege-Technik handelt es sich um ein trennendes und umformendes Fertigungsverfahren, bei dem ein oder mehrere Halbzeuge, wie z.B. Metallbänder oder Metalldrähte, auf einem einzigen Automaten, der Stanz-Biege-Maschine, zu einem fertigen Produkt verarbeitet werden. \cite{Hoermann2013}

Da den Stanz-Biege-Maschinen seit 1966 ein Baukastensystem zugrunde liegt, besitzen sie eine Vielzahl an Bearbeitungsvariationen und Umformmöglichkeiten und es wurden immer weitere Prozessschritte in das Bearbeitungssystem integriert. Neben den Stanz-Biege-Maschinen, die das Kerngeschäft der Firma Bihler bilden, werden deshalb die Montagemaschinen immer wichtiger. Auf ihnen können in Kombination mit Stanz-Biege-Maschinen unter dem Einsatz von weiteren Bearbeitungsschritten, wie Montieren, Schweißen, Gewindeschneiden, Messen, Verpacken und Beschriften, ganze Herstellungsprozesse realisiert werden. Es eröffnen sich Möglichkeiten zu einer Komplettfertigung von immer komplexeren Bauteilen und Baugruppen  bei minimalem Material- und Energieeinsatz, was den Anforderungen der Zulieferindustrie entgegenkommt.

Die Bihler Technologie wird in verschiedensten Industriezweigen eingesetzt. So findet man die Maschinen und die darauf produzierten Produkte laut \cite{Hoermann2013} in der Automobilindustrie, Elekto- und Elektronikindustrie, Medizintechnik, Registratur- und Verbindungstechnik, Schmuckindustrie, Federn- und Drahtindustrie, Kommunikationstechnik, Eisen-, Blech- und Metallwarenindustrie sowie der Umwelttechnik.


Historisch erfolgte der Antrieb in den meisten Maschinentypen über ein von einem Elektromotor angetriebenens Großrad. Die auf der Maschine eingesetzten Aggregate werden entweder direkt oder über eine Übersetzung durch das Großrad angetrieben. Durch diese zwangsangetriebenen Aggregate können alle Bearbeitungsschritte ohne zusätzliche Steuerung realisiert werden. Nachteilig ist hier, dass Bewegungen, die nicht innerhalb einer Maschinenumdrehung durchgeführt werden können, aufwendig durch \gls{NC}-gesteuerte Hilfsaggregate oder durch Pneumatik ausgeführt werden müssen.

Die Steuerung der mechanischen Maschinen erfolgt über Kurvenscheiben. Hierbei wird die rotatorische Bewegung des Großrades auf das Aggregat übertragen und über die Kurvenscheiben in eine Linearbewegung umgewandelt. Kurvengetriebe sind sehr kompakt und zuverlässig und ermöglichen einen schnellen Werkzeugwechsel, da nur die Kurvenscheibe ausgetauscht werden muss. Dennoch sind sie bei den immer kleiner werdenden Losgrößen heutzutage nicht flexibel genug, um schnell und kostengünstig zu produzieren.

Schon im Jahr 2000 wurde das erste komplett elektronisch gesteuerte \gls{CNC}-Umformsystem entwickelt. So wurden unter anderem die mechanisch angetriebenen Schlittenaggregate durch elektrisch direkt angetriebene und elektronisch gesteuerte \glspl{NCA} (NCAs) ersetzt. Der Durchbruch auf dem Markt gelang der NC-Technik im Hause Bihler mit den sehr flexiblen BIMERIC Maschinen. Diese sind durch das Baukastensystem besonders gut für Montageaufgaben geeignet. Mit den neu entwickelten GRM-NC und RM-NC Maschinen stehen Aggregate zur Verfügung, die die klassisch mechanisch gesteuerten Stanz-Biege-Automaten ersetzen können. 


Eines der Hauptelemente der NC gesteuerten Maschinen sind hierbei die Numerical Controlled Aggregate (NCAs). Diese ersetzen die kurvengesteuerten Aggregate und sind für die Ausführung von Werkzeugbewegungen verantwortlich. Die benötigte Stückzahl an NCAs ist in den letzten Jahren stark gestiegen, da sie sich als für den Kunden vorteilhaft in der Produktion von komplexen Bauteilen herausgestellt haben. Die Anforderungen an die NCAs im Einsatz sind sehr hoch, da sie ständig und mit hoher Belastung laufen müssen. Zudem handelt es sich bei den NCAs um eine für den Stanz-Biegebereich neue Technologie, die sich in der Weiterentwicklungs- und Optimierungsphase befindet.




Der Einsatz der NCA Technologie im Hause Bihler geht auf die Konstruktion und Entwicklung dieser Aggregate im eigenen Haus zurück. Es werden Prototypen gebaut, getestet und weiterentwickelt. Die NCAs werden inzwischen in Serie am Standort Füssen gefertigt und montiert. Für die verschiedenen Einsatzzwecke gibt es verschiedene Varianten, die sich im Verfahrweg, der Verfahrgeschwindigkeit, den Kräften und daraus bedingt in der Größe und dem Gewicht unterscheiden.

Um den hohen Qualitätsansprüchen der Firma Bihler gerecht zu werden, werden die NCAs, nachdem sie fertig montiert sind, im Werk einem Testlauf unterzogen. Neben dem eigentlichen Testen dient der Testlauf auch dazu, die Aggregate einlaufen zu lassen. Nach dem Testlauf werden die NCAs direkt in neue Maschinen verbaut oder als Ersatzgeräte an Kunden ausgeliefert. Jedoch treten bei Kunden immer wieder Fehler auf, so dass die schadhaften NCAs zurückgeliefert und durch die Firma Bihler repariert bzw. ersetzt werden müssen.


\section{Aufgabenstellung}

Dies ist der konkrete Anlass dafür, sich mit dem Testablauf der NCAs auseinanderzusetzen. Denn einwandfrei funktionierende NCAs sind grundlegend für die Produktion beim Kunden und es kann bei ihm zum Stillstand der Maschine und dadurch bedingt einem erheblichen Produktionsausfall kommen, wenn sie fehlerhaft sind. Deshalb müssen durch das Testen vor dem Einbau in eine Maschine möglichst weitgehend alle möglichen Fehlerquellen ausgeschlossen werden. Bei vom Kunden zurückgelieferten Aggregaten wurde zum Beispiel Spiel festgestellt. Dieses Spiel ist durch den bisherigen Testlauf nicht erkannt worden.



% Konkrete Anlässe dafür, sich mit dem Testablauf auseinanderzusetzen, ergeben sich daraus, dass bei den Kunden Fehler auftreten und die schadhaften NCAs zurückgeliefert und durch die Firma Bihler repariert bzw. ersetzt werden müssen. Da  einwandfrei funktionierende NCAs grundlegend für die Produktion beim Kunden sind und es bei ihm zum Stillstand der Maschine und dadurch bedingt einem erheblichen Produktionsausfall kommen kann, wenn sie fehlerhaft sind, müssen durch das Testen vor dem Einbau in die Maschine möglichst weitgehend alle möglichen Fehlerquellen ausgeschlossen werden. Bei vom Kunden zurückgelieferten Aggregaten wurde zum Beispiel Spiel festgestellt. Dieses Spiel ist durch den bisherigen Testlauf nicht erkannt worden.

So ist der derzeitige Testablauf nach Ansicht der Firma Bihler nicht optimal gestaltet und es ist Aufgabe dieser Diplomarbeit, den Prüfprozess der NCAs in der Serienfertigung zu analysieren und gegebenfalls zu überarbeiten. Falls nötig, sollen neue Prüfkriterien bzw. Testmerkmale gefunden und Testmöglichkeiten entwickelt werden. 

Die Arbeit wird in der Abteilung Serienbetreuung der Firma Bihler am Standort Füssen durchgeführt. Dort können sowohl die Erfassung und Auswertung des derzeitigen Testablaufs, als auch das theoretische Erarbeiten  möglicher Testmodelle sowie das Testen verschiedener NCAs mit verschiedenen Testmodellen am bereits vorhandenen Prüfstand der NCAs vorgenommen werden.


In dieser Arbeit wird schwerpunktmäßig mit den konkreten Werten eines bestimmten NCA-Typs, mit der Artikelnummer 100-54-0700.5 (NCA 4 / 120.19000 P=10 vgl. \cite{Riedle2015}), gerechnet. Die hier dargestellten Methoden sind jedoch für alle NCAs gültig. Auf mögliche Unterschiede und Probleme wird in den einzelnen Kapiteln eingegangen. Aufgrund der Größe und des Handlings sind die Varianten der NCA7 kein Bestandteil dieser Arbeit.








