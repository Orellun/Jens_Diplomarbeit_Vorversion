\chapter*{Zusammenfassung}\addcontentsline{toc}{chapter}{Zusammenfassung}

Das Labor für Systemtechnik entwickelt, produziert und betreibt seit 2010 im Rahmen sowohl von studentischen Projekten als auch von Forschungsprojekten verschiedene unbemannte Flugsysteme.
Der Schwerpunkt liegt hier bei automatisch gesteuerten Flächenflugzeugen.
Das Abfluggewicht liegt zwischen 2 und 15 kg sowie Spannweiten zwischen 1,5 m und 5 m.

-->> Hier kurze Erwähnung AUVSI 

-->> Komplexes System genötigt verschiedenste Elektronik für Antrieb Aktuierung/Lenkung Funkkontakt Energiemanagement etc.

-->> Dafür wird in dieser Arbeit ein Elektrisches Funktionskonzept erstellt und die Anforderungen formuliert

-->> Es werden Baugruppen entwickelt, die die jeweiligen Subanforderungen erfüllen sollen

-->> Test von Subkomponenten (Ideale Diode)

-->> Elektronischer Gesamtaufbau entsteht und wird erklärt

-->> Elektronikerprobung wird beschrieben und Messdaten gezeigt (so.)

-->> Flugversuch Praxistest ---> Wettbewerbseinsatz und Wettbewerbsergebnis

