\chapter*{Zusammenfassung}\addcontentsline{toc}{chapter}{Zusammenfassung}

Das Labor für Systemtechnik entwickelt, produziert und betreibt seit 2010 im Rahmen sowohl von studentischen Projekten als auch von Forschungsprojekten verschiedene unbemannte Flugsysteme.
Der Schwerpunkt liegt hier bei automatisch gesteuerten Flächenflugzeugen.
Deren Abfluggewicht liegt zwischen 2 und 15 kg sowie Spannweiten zwischen 1,5 m und 5 m.

Auch werden diese Flugsysteme seit 2015 im Studentischen AUVSI SUAS Wettbewerb eingesetzt. Dabei konnten zunehmend gute Platzierungen erzielt werden.
Die Systeme benötigen eine Vielzahl elektrischer Funktionen für ihre Steuerung, Lenkung, erhalten einer Funkverbindung und Management ihres Energieflusses.
Um eine Einbeziehung aller notwendigen Funktionen sicherzustellen wird ein Abstraktes Anforderungsprofil erstellt und die Umsetzung in zwei Baugruppen der Leistungselektronik und Autopilotenelektronik definiert.

Es wird eine platzsparende Platinenbasierte Ausführung aller Funktionen in Hardware realisiert. Dafür kommen neben zugekauften Baugruppen in der Spannungswandlung, hauptsächlich das Prinzip der Idealen Diode im Leistungspfad zum Einsatz. Außerdem wird eine Priorisierung des Leistungspfades implementiert.

Diese Funktionen in Kombination mit weiteren Sicherheitsmaßnahmen, wie einer eineindeutigen Steckverbinder Wahl, ermöglicht einen sicheren und zuverlässigen Betrieb des Flugsystems sowohl für den SUAS Wettbewerb als auch weitere Forschungsaufträge.

Zur Sicherstellung der korrekten Funktion und Feststellung der Leistungsgrenzen werden die Idealen Dioden Vermessen und untereinander Verglichen. Die Daten zeigen eine Einsatzbereich der Verbesserten Generation bis zu 30 Ampere Dauerstrom.

Für die Weitere Entwicklung zeigen sich ein neues Batteriesystem im Zylinderformat und der Einsatz neuer Sensoren auf CAN Bus Basis als vielversprechend.

