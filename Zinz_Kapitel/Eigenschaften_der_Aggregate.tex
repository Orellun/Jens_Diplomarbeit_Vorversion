\chapter{Funktionsspezifische Eigenschaften der NC-Aggregate} \label{Eigenschaften_der_Aggregate}



\blockquote{Prüfen nennt man die Feststellung ob ein Prüfgegenstand eine oder mehrere vereinbarte oder vorgeschriebene Bedingungen erfüllt.} \cite{pesch2013messen}


%Test: ein Test ist eine Maßnahme, welche dazu geeignet ist, spezifische Eigenschaften eines Produktes, eines Messmittels, von Material, Ausrüstung, Organismen, physikalische Begebenheiten, Prozessen oder Abläufen nach definierten Prozeduren zu überprüfen.


Um prüfen zu können, müssen erst die vereinbarten oder vorgeschriebenen Bedingungen bekannt sein. Deshalb ist die Voraussetzung, um Prüfkriterien für die NCAs zu entwickeln, dass man sich mit deren Eigenschaften auseinandersetzt. 

Laut Duden ist eine Eigenschaft  ein \textquote{zum Wesen einer Person oder Sache gehörendes Merkmal; charakteristische [Teil]beschaffenheit oder [persönliche, charakterliche] Eigentümlichkeit}. \cite{Duden_Eigenschaft}


Die technisch komplex aufgebauten NCAs besitzen viele bisher nicht näher bestimmte Eigenschaften. Diese Eigenschaften ergeben sich aus dem Zusammenspiel aller Komponenten.  Die vielen Komponenten (vgl. Kapitel~\ref{cha:Beschreibung_der_NCAs}) wirken zusammen, um die Hauptfunktionen der NCAs zu verwirklichen. Die Hauptfunktionen der NCAs sind, dass sie eine Linearbewegung und eine Kraft zur Verfügung stellen. Darüber hinaus müssen weitere Nebenfunktionen wie Kühlung, Schmierung, Dichtigkeit und eine rechtzeitig auslösende Haltebremse gewährleistet sein, um die Hauptfunktionen erfüllen zu können. Die Funktionen sind in der Übersicht von Abbildung~\ref{fig:Uebersicht_ueber_die_wichtigsten_Funktionen_von_NCAs} aufgeführt.


\begin{figure}[H]
\center
\fbox{
\begin{minipage}[c]{0.8\textwidth}
\vspace{5pt}
\begin{enumerate}
 \item Hauptfunktionen
 \begin{itemize}
    \item Kraft
    \item Bewegungsprofil (Wegverlauf über Zeit)
 \end{itemize}
 \item Nebenfunktionen
 	\begin{itemize}
	 \item Kühlung
	 \item Schmierung
     \item Dichtigkeit
     \item Haltebremse
    \end{itemize}
\end{enumerate}
\par\vspace{2pt}
\end{minipage}}
\caption{Übersicht über die wichtigsten Funktionen von NCAs}
\label{fig:Uebersicht_ueber_die_wichtigsten_Funktionen_von_NCAs}
\end{figure}






\section{Beschreibung der funktionsspezifischen Eigenschaften}

Damit die NCAs ihre Hauptfunktionen erfüllen können, müssen viele Einzelbedingungen erfüllt sein. Besonders wichtig in diesem Zusammenhang ist, dass die Ausgangsmaterialien die geforderte Qualität haben, dass die Geometrie der Einzelbauteile  den jeweils vorgegebenen Toleranzen entspricht und dass die Ausrichtung der montierten Bauteile den Anforderungen entspricht. Alle Schrauben und Spannmuttern müssen nach den jeweiligen Vorgaben angezogen und je nach Bedarf gesichert sein, um ihre Funktionen zu erfüllen. Um den Kundenanforderungen zu entsprechen, müssen die NCAs optisch einen ansprechenden Eindruck machen. Alle Bauteile müssen vorhanden sein und es dürfen keine sichtbaren Schäden wie Kratzer und Schrammen vorliegen. 

Die elektrischen Komponenten müssen von der Steuerung erkannt werden und auf die Signale der Steuerung erwartungsgemäß reagieren. Außerdem müssen die elektrischen Komponenten die an sie gestellten Aufgaben erledigen. So müssen die Sensoren die Messgrößen richtig bestimmen und richtig an die Steuerung weitergeben und die Haltebremse muss bei angelegter Spannung geöffnet sein.

Um die Funktion des Aggregates dauerhaft sicherzustellen, muss die Schmierung des Aggregates funktionieren. Der Austritt von Schmiermittel in dafür nicht vorgesehene Regionen wird durch funktionierende Dichtungen verhindert. Insbesondere ist das Austreten von Schmiermittel aus dem Aggregat in die Maschine zu verhindern. Eine weitere wichtige Bedingung ist eine funktionierende Kühlung. Die Kühlung des Motors und eventuell des Aggregates dient dazu, die entstandene Entropie abzuführen und die Motortemperatur konstant zu halten. Hierzu ist ein entsprechender Durchfluss an Kühlmittel notwendig. Allerdings darf das Kühlmittel nur in den dafür vorgesehenen Bahnen laufen, da es sonst das restliche Aggregat zerstört.

Weiterhin muss beim Drehen der Gewinderollenspindel die Pinole leichtgängig ein- und ausfahren. Hierbei darf es zu keinem übermäßigen Verschleiß an den Laufflächen kommen. Auch muss eine hohe Positionier- und Wiederholgenauigkeit bei allen möglichen Belastungsarten des Aggregates gegeben sein. 

Letztendlich müssen die durch die Konstruktion vorgegebenen Leistungswerte der Geschwindigkeit, der Beschleunigung, der Spitzenhaltekraft (kurzzeitig aufbringbare Kraft bis der Motor überhitzt) und der Dauerhaltekraft (dauerhaft aufbringbare Kraft ohne dass der Motor überhitzt) erreicht werden.

Zur Zeit bestehen keine Standards für die funktionsspezifischen Eigenschaften, anhand derer die NCAs beurteilt werden könnten.



\section{Störungen der funktionsspezifischen Eigenschaften}

Wie bei jedem technischen Produkt treten immer wieder Probleme dadurch auf, dass es die geforderten Eigenschaften nicht aufweist und dadurch seine Funktion nicht erfüllt.  Die auftretenden Störungen können verschiedene Ursachen haben. 

\subsection{Entdeckungspunkte von Funktionsstörungen} \label{cha:Entdecker_von_Funktionsstoerungen}


%\colorbox{orange}{Überschriftsvorschlag: Bereiche zum Entdecken von Funktionsstörungen, Bereiche in denen Funktionsstörungen entdeckt werden}



Von der Entwicklung bis zum Einsatz beim Endkunden sind verschiedenste Bereiche der Firma Bihler mit den NCAs konfrontiert. Fehler bzw. Funktionsstörungen werden in allen mit den NCAs in Kontakt kommenden Abteilungen und den mit den NCAs konfrontierten Personenkreisen entdeckt. Dies zeigt die tabellarische Übersicht in Abbildung~\ref{fig:Uebersicht_ der_mit_Funktionsstoerungen_konfrontierten_Bereiche}.

Die Konstruktionsabteilung ist mit der Erstentwicklung bzw. der Optimierung und Weiterentwicklung der NCAs betraut. Probleme, die auf Konstruktionsmängeln beruhen, werden von allen anderen Abteilungen wie z.B. Versuch, Vormontage, Werkzeugbau Endmontage und Customer Support an sie zurückgemeldet. Dort müssen Lösungen entwickelt werden. Die Hauptaufgabe der Versuchsabteilung ist es,  Fehler schon in der Entwicklungsphase zu finden und die von der Konstruktion bereitgestellten Parameter zu verifizieren. Man kann die Versuchsabteilung auch als 'ersten Anwender' sehen, da dort die in der Fertigung und in der Vormontage hergestellten Prototypen neben einer allumfassenden Überprüfung auch Dauerbelastungstests unterzogen werden. 

Haben die NCAs die Tests bestanden, werden sie für die Serienfertigung freigegeben. In der Fertigung werden die notwendigen Bauteile gefertigt und jeweils geprüft, d.h. Fehler werden schon im Anfangsstadium entdeckt und behoben. Die Abteilung Qualitätssicherung unterstützt dies, indem sie stichprobenartig Einzelteile vermisst. Einen entscheidenden Baustein bei dem Erkennen von Fehlern stellt die Abteilung Vormontage dar. Die an sie gelieferten Einzelbauteile werden dort zur Baugruppe NCA zusammenmontiert und somit eventuelle Fertigungsfehler festgestellt. Darüber hinaus ist dort auch der Prüfstand für die NCAs angesiedelt, auf dem mit Testläufen festgestellt werden kann, ob die Baugruppe funktioniert. Möglichst viele konstruktive Fehler und Fertigungsfehler sollen hier erkannt werden. Neben eigentlichen Testläufen wird der Prüfstand auch genutzt, um die NCAs einlaufen zu lassen, indem sie ohne Belastung langsam gefahren werden.

Die getesteten NCAs werden in der Abteilung Maschinenbau Endmontage auf die fertige Maschine montiert. Um festzustellen, ob die komplexe Maschine funktionstüchtig ist, wird sie einem mehrstündigen Testlauf unterworfen, indem sie ohne Belastung betrieben wird. Somit durchlaufen die NCAs noch einmal einen Testlauf. Während danach einige Maschinen an den Endkunden ausgeliefert werden, werden andere in der Abteilung Werkzeugbau Endmontage mit Werkzeugen bestückt. Beim Einrichten und Testen des Werkzeugs werden auch die NCAs für diesen konkreten Einsatz automatisch unter Belastung getestet.  


Den Einsatz der Maschinen beim Endkunden und somit den Einsatz der NCAs könnte man als Dauertest unter Praxisbedingungen auffassen. Erst hier wird die noch neue Technologie der NCAs einer extrem starken und dauerhaften Belastung ausgesetzt. Deshalb kumulieren hier sämtliche Fehler, die bei der Konstruktion oder Herstellung unterlaufen sind, und führen meist zu einem Totalausfall des NCAs. Für die schadhaften NCAs ist die Abteilung Customer Support zuständig, denn sie stellt den Kundenkontakt her. Sie  versucht den Fehler vor Ort zu beheben oder bekommt das NCA als Rückläufer in die Abteilung geliefert. Bei entsprechenden Problemen meldet sie diese an die Konstruktionsabteilung weiter.


\begin{figure}[H]
\center
\fbox{
\begin{minipage}[c]{0.8\textwidth}
\vspace{5pt}
\begin{itemize}
 \item Konstruktion 
 \item Versuch
 \item Fertigung
 \item Qualitätssicherung
 \item Vormontage
 \item Maschinenbau Endmontage
 \item Werkzeugbau Endmontage
 \item Customer Support
 \item Endkunde 
\end{itemize}
\par\vspace{2pt}
\end{minipage}}

\caption{Übersicht der mit Funktionsstörungen konfrontierten Bereiche}
\label{fig:Uebersicht_ der_mit_Funktionsstoerungen_konfrontierten_Bereiche}
\end{figure}




\subsection{Ursachen von Funktionsstörungen}\label{cha:Ursachen_von_Funktionsstoerungen}

Um Prüfkriterien zum Erkennen von Funktionsstörungen entwickeln zu können, muss man in Frage kommende Ursachen kennen. Funktionsstörungen können in unterschiedlichsten Stadien im Laufe des Produktlebenszyklus der NCAs entstehen und sie können vielfältigste Ursachen haben, wie die Übersicht in Abbildung~\ref{fig:Ursachen_von_Funktionsstoerungen} zeigt.

\begin{figure}[H]
\center
\fbox{
\begin{minipage}[c]{0.8\textwidth}
\vspace{5pt}
\begin{itemize}
 \item konstruktive Fehler
 \item Fertigungsfehler der Einzelkomponenten
 \item Fehler bei der Montage 
 \item fehlerhafte Programmierung
 \item nicht bestimmungsgemäßer Gebrauch
 \item nicht bestimmungsgemäße Instandhaltung
 \item Verschleiß und Alterungserscheinungen
\end{itemize}
\par\vspace{2pt}
\end{minipage}}
\caption{Ursachen von Funktionsstörungen}
\label{fig:Ursachen_von_Funktionsstoerungen}
\end{figure}


Sehr leicht können Fehler in der Konstruktion entstehen. Konstruktive Fehler sind besonders durch die hohen Kosten, die sie verursachen, ein Problem. Zu neuen konstruktiven Fehlern kann es immer wieder kommen, wenn die NCAs weiterentwickelt werden. Trotz des hohen Niveaus der Fertigung kann es zu Fehlern bei der Produktion der Einzelkomponenten kommen. Werden diese durch die fertigungseigene Prüfung oder in der Qualitätssicherung nicht erkannt, führt dies meist in der Montage oder später zu Problemen. Und auch die Montage der vielen einzelnen Komponenten zum kompletten NC-Aggregat kommt als Fehlerquelle in Frage. Einige Montagefehler können bereits durch sachgemäße Konstruktion ausgeschlossen werden.

Ein weiteres Feld sind Fehler in der Programmierung. So passt eventuell die Parametrierung nicht zu den verwendeten Bauelementen. Auch kann ein Programmierfehler mechanische oder elektrische Fehler vortäuschen. Zu einem nicht bestimmungsgemäßen Gebrauch mit den entsprechenden Folgen kann es beim Anwender kommen, indem er die Aggregate z.B. überlastet, überhitzt, nicht genug kühlt oder nicht genug schmiert. Auch eine nicht bestimmungsgemäße Instandhaltung und Wartung ist denkbar, die die Funktion einschränkt oder gänzlich stört. Besonders am Ende ihrer Lebenszeit kommt es bei den NCAs vermehrt zu Problemen durch Verschleiß und Alterungserscheinungen. Nicht außer acht gelassen werden darf auch, dass es zu einer Kombination verschiedener Fehler kommen kann, was das Identifizieren der tatsächlichen Fehlerursache erschwert. 

Viele Fehler werden zu unterschiedlichen Zeiten der Herstellung, Testung und Nutzung der Aggregate entdeckt, häufig leider jedoch erst beim Endkunden. In Kapitel~\ref{cha:Bekannte_Fehler_und_Probleme} sind einzelne bekannt gewordene Fehler und ihre möglichen Ursachen näher erläutert.




\subsection{Dokumentation von Funktionsstörungen}\label{cha:Dokumentation_von_Funktionsstoerungen}

Über die im Haus aufgetretenen Fehler gibt es keine Aufzeichnungen. Diese Fehler werden entweder durch die Vormontage selbst, bei der Endmontage der Maschine in Füssen, in der Werkzeugbau Endmontage in Halblech oder in den weiteren, unter Kapitel~\ref{cha:Entdecker_von_Funktionsstoerungen} aufgeführten, Abteilungen entdeckt. Viele Fehler werden hier bereits gefunden. Nachdem viele Fehler während des Produktions- und Montageprozesses gefunden werden, können sie schon frühzeitig behoben werden.





Da eine systematische Erfassung der im Haus aufgetretenen Fehler nicht stattfindet, sind verifizierbare Aussagen über deren Art, Häufigkeit und wie gravierend sie sind, nicht möglich. Man kann manche Fehler deshalb nicht strukturiert dauerhaft beseitigen und nutzt die vorhandenen Kapazitäten nicht effektiv. Weiterhin bedeutet die fehlende Dokumentation, dass das Wissen über die NC-Achsen sehr stark bei den einzelnen beteiligten Mitarbeitern liegt. Dies führt bei einem Ausscheiden des Mitarbeiters aus der Firma zu großen Problemen, denn ein neuer Mitarbeiter kann auf keinen dokumentierten Erfahrungsschatz zurückgreifen. Des Weiteren beginnt bei jedem neu auftauchenden Problem die Suche nach der möglichen Fehlerursache wieder von vorne.



\subsection{Auswertung des Reklamationsmanagements}\label{cha:Auswertung_des_Reklamationsmanagements}




Bei der Firma Bihler werden Reklamationen von Kunden mit dem Reklamationsmanagementsystem CAQ=QSYS®  erfasst und alle relevanten Informationen dokumentiert. Mit dem Reklamationsmanagementsystem werden nur Aggregate erfasst, die bereits dem Kunden ausgeliefert wurden.



Die Auswertung der Reklamationen ist durch \cite{Guggemos2015} erfolgt. Die Daten sind innerhalb des Zeitraumes vom 01.08.2014 bis zum 15.05.2015 entstanden. Vor dem August 2014 wurden keine Daten zu den Reklamationen erfasst.

Derzeit werden hauptsächlich NCA 5 reklamiert (vgl. Abbildung~\ref{fig:Anzahl_der_Reparaturen}). Es liegen keine konkreten Zahlen vor, welche NCAs wann ausgeliefert wurden. So ist nicht klar, wie viele der einzelnen Modelle anteilmäßig an die Kunden ausgeliefert wurden und wie lange sich die einzelnen Modelle jeweils im Einsatz befinden. Es wird vermutet, dass NCA 5 die am weitesten verbreitete Variante ist und sich auch schon am längstem im Umlauf befindet.

\begin{figure}[H]



\begin{tikzpicture} 
 \begin{axis}[ 
  width=\textwidth,height=0.3\textheight,
  ybar,
  ymin=0,
  ymax=33,
  bar width=45pt,
  legend style={at={(0.5,-0.2)}, 
   anchor=north,legend columns=-1}, 
   ymajorgrids=true,
  ylabel={Anteil in Prozent},
  ytick={0,8.25,16.5,24.75,33},
  yticklabel={\pgfmathparse{\tick*100/33}\pgfmathprintnumber{\pgfmathresult}},
  symbolic x coords={NCA 5,NCA 4,NCA 3,NCA 2}, 
  xtick=data, 
  nodes near coords, 
    nodes near coords align={vertical}, 
  x tick label style={font=\small,anchor=north,text width=3cm,align=center}, 
   enlarge x limits=.3,
    ]
  \addplot 
    [fill=Bihler1]
        coordinates {
        (NCA 5,24)
        (NCA 4,3)
        (NCA 3,2)
        (NCA 2,4)
  }; 
 \end{axis} 
\end{tikzpicture}
\caption{Anzahl der Reklamationen im Zeitraum 01.08.2014 bis 15.05.2015}
\label{fig:Anzahl_der_Reparaturen}
\end{figure}



In den Abbildungen~\ref{fig:Ursachen_der_Reparaturen_beim_NCA_5},~\ref{fig:Ursachen_der_Reparaturen_beim_NCA_4},~\ref{fig:Ursachen_der_Reparaturen_beim_NCA_3} und ~\ref{fig:Ursachen_der_Reparaturen_beim_NCA_2} sind die Ursachen für die Reklamationen der einzelnen NCA Typen aufgezeigt. Aufgrund der geringen Anzahlen von Reklamationen für die NCA Typen 2, 3 und 4 lassen sich differenzierte Aussagen über die Arten der Funktionsstörungen der einzelnen Typen nicht treffen.




\begin{figure}[H]
\begin{tikzpicture} 
 \begin{axis}[ 
  width=\textwidth,
  height=0.2\textheight,
  ybar,
  ymin=0,
  ymax=4,
  bar width=55pt,
  legend style={at={(0.5,-0.2)}, 
   anchor=north,legend columns=-1}, 
   ymajorgrids=true,
  ylabel={Anteil in Prozent},
  yticklabel={\pgfmathparse{\tick*100/4}\pgfmathprintnumber{\pgfmathresult}},
  symbolic x coords={Schmieröl im Motorgeber,Pinole in Gehäuse gefressen,allg. Fehler beim Motorgeber (Fehler b. Lieferanten)}, 
  xtick=data, 
  nodes near coords, 
    nodes near coords align={vertical}, 
  x tick label style={font=\small,anchor=north,text width=4cm,align=center},
  enlarge x limits=.3,
    ] 
  \addplot 
    [fill=Bihler1]
        coordinates {
        (Schmieröl im Motorgeber,2) 
        (Pinole in Gehäuse gefressen,1)
        (allg. Fehler beim Motorgeber (Fehler b. Lieferanten),1) 
  }; 
 \end{axis} 
\end{tikzpicture}
\caption{Ursachen der Reparaturen beim NCA 2}
\label{fig:Ursachen_der_Reparaturen_beim_NCA_2}
\end{figure}

\begin{figure}[H]

\begin{tikzpicture} 
 \begin{axis}[ 
  width=\textwidth,
  height=0.15\textheight,
  ybar,
  ymin=0,
  ymax=2,
  bar width=70pt,
  legend style={at={(0.5,-0.2)}, 
   anchor=north,legend columns=-1}, 
   ymajorgrids=true,
  ylabel={Anteil in Prozent},
  yticklabel={\pgfmathparse{\tick*100/2}\pgfmathprintnumber{\pgfmathresult}},
  ytick={0,1,2,3,4},
  symbolic x coords={Motor: durchgebrannt (Überlast),Motor: Kühlsystem undicht}, 
  xtick=data, 
  nodes near coords, 
    nodes near coords align={vertical}, 
  x tick label style={font=\small,anchor=north,text width=3cm,align=center},
  enlarge x limits=.7,
    ] 
  \addplot 
    [fill=Bihler1]
        coordinates {
        (Motor: durchgebrannt (Überlast),1) 
        (Motor: Kühlsystem undicht,1) 
  }; 
 \end{axis} 
\end{tikzpicture}
\caption{Ursachen der Reparaturen beim NCA 3}
\label{fig:Ursachen_der_Reparaturen_beim_NCA_3}
\end{figure}

\begin{figure}[H]

\begin{tikzpicture} 
 \begin{axis}[ 
  width=\textwidth,
  height=0.2\textheight,
  ybar,
  ymin=0,
  ymax=3,
  bar width=70pt,
  legend style={at={(0.5,-0.2)}, 
   anchor=north,legend columns=-1}, 
   ymajorgrids=true,
  ylabel={Anteil in Prozent},
  ytick={0,0.75,1.5,2.25,3},
  yticklabel={\pgfmathparse{\tick*100/3}\pgfmathprintnumber{\pgfmathresult}},
  symbolic x coords={Schmieröl im Motorgeber,allg. Fehler beim Motorgeber (Fehler b. Lieferanten)}, 
  xtick=data, 
  nodes near coords, 
    nodes near coords align={vertical}, 
  x tick label style={font=\small,anchor=north,text width=4cm,align=center},
  enlarge x limits=.7,
    ] 
  \addplot 
    [fill=Bihler1]
        coordinates {
        (Schmieröl im Motorgeber,2) 
        (allg. Fehler beim Motorgeber (Fehler b. Lieferanten),1) 
  }; 
 \end{axis} 
\end{tikzpicture}
\caption{Ursachen der Reparaturen beim NCA 4}
\label{fig:Ursachen_der_Reparaturen_beim_NCA_4}
\end{figure}



\begin{figure}[H]

\begin{tikzpicture} 
 \begin{axis}[ 
  width=0.8\textwidth,height=0.3\textheight,
  ybar,
  ymin=0,
  ymax=7,
  bar width=20pt,
  legend style={at={(0.5,-0.2)}, 
   anchor=north,legend columns=-1}, 
   ymajorgrids=true,
  ylabel={Anteil in Prozent},
  yticklabel={\pgfmathparse{\tick*100/24}\pgfmathprintnumber{\pgfmathresult}},
  ytick={0,1.2,2.4,3.6,4.8,6},
  symbolic x coords={allg. Fehler AMO-Messsystem,Pinole in Gehäuse gefressen,Kühlkreislauf zugesetzt,Defekt am Motor (Fehler b. Lieferanten),axiales Spiel,Montagefehler Spannsatz,Gleitstein beschädigt (auf Block gefahren)}, 
  xtick=data, 
  nodes near coords, 
    nodes near coords align={vertical}, 
  x tick label style={font=\small,rotate=30,anchor=east}, 
    ] 
  \addplot 
    [fill=Bihler1]
        coordinates {
        (allg. Fehler AMO-Messsystem,6) 
        (Pinole in Gehäuse gefressen,5) 
        (Kühlkreislauf zugesetzt,5) 
        (Defekt am Motor (Fehler b. Lieferanten),3) 
        (axiales Spiel,2)
        (Montagefehler Spannsatz,2)
        (Gleitstein beschädigt (auf Block gefahren),1)
  }; 
 \end{axis} 
\end{tikzpicture}
\caption{Ursachen der Reparaturen beim NCA 5}
\label{fig:Ursachen_der_Reparaturen_beim_NCA_5}
\end{figure}









Wenn man die Reklamationen aller 4 NCA Typen betrachtet, sind die elektrischen und die mechanischen  Komponenten in etwa zu gleichen Anteilen (16 elektrische zu 17 mechanischen Ausfällen) für die Funktionsstörungen aller erfassten NCAs verantwortlich. Fehler im Messsystem, Fehler im Motor oder Fehler im Zusammenhang mit dem Motorgeber werden den elektrischen Komponenten zugerechnet. Als mechanische Fehler sind Fressen der Pinole im Gehäuse, zugesetzter Kühlkreislauf, axiales Spiel und Montagefehler einzuordnen.



Einige der Funktionsstörungen sind auf fehlerhafte Komponenten von Zulieferern zurückzuführen. Diese sind in den Abbildungen~\ref{fig:Ursachen_der_Reparaturen_beim_NCA_5}, unter Defekt am Motor (Fehler b. Lieferanten), \ref{fig:Ursachen_der_Reparaturen_beim_NCA_4} und in ~\ref{fig:Ursachen_der_Reparaturen_beim_NCA_2} unter allg. Fehler beim Motorgeber (Fehler b. Lieferanten) zu finden. Alle hier aufgeführten Fehler sind so gravierend, dass sie zu einem Totalausfall der NCAs geführt haben. Fehler, die nicht zu einem Komplettausfall der NCAs führen, werden mit dem jetzigen Reklamationssystem nicht erfasst. Die konkret aufgetretenen Funktionsstörungen der reklamierten Aggregate sind in Kapitel~\ref{cha:Bekannte_Fehler_und_Probleme} näher ausgeführt.




%Wie in den Abbildungen~\ref{fig:Ursachen_der_Reparaturen_beim_NCA_5}, Defekt am Motor (Fehler b. Lieferanten), \ref{fig:Ursachen_der_Reparaturen_beim_NCA_4} und~\ref{fig:Ursachen_der_Reparaturen_beim_NCA_2} allg. Fehler beim Motorgeber (Fehler b. Lieferanten) zu sehen.






\subsection{Bekannte Funktionsstörungen der NC-Aggregate} \label{cha:Bekannte_Fehler_und_Probleme}






Nach den theoretischen Überlegungen zu den Bereichen, in denen Funktionsstörungen entdeckt werden können (vgl. Kapitel~\ref{cha:Entdecker_von_Funktionsstoerungen}) und zu deren möglichen Ursachen (vgl. Kapitel~\ref{cha:Ursachen_von_Funktionsstoerungen}) hat sich aufgrund der Reklamationen (vgl. Kapitel~\ref{cha:Auswertung_des_Reklamationsmanagements}) im Betriebsalltag herausgestellt, dass der Endkunde mit massiven Störungen konfrontiert ist. Und auch im Herstellungsprozess stößt man auf Fehler, die weitreichende Folgen haben können. Jegliche Fehler sollen durch einen entsprechenden Testprozess in Grenzen gehalten werden, was nur möglich ist, wenn man die Funktionstörungen und eventuell sogar ihre Ursachen kennt.

Zu Anzahl, Art und Umfang an Fehlern, die innerhalb der Firma Bihler entdeckt werden, und zu deren Ursachen sind keine Daten vorhanden, wie in Kapitel~\ref{cha:Dokumentation_von_Funktionsstoerungen} ausgeführt. Deswegen sind diese Fehler hinsichtlich ihrer Häufigkeit oder Konsequenzen nicht bewertbar.

Auch sind quantitative Aussagen bezüglich Fehlern, die beim Kunden vorkommen, nur eingeschränkt möglich, denn die Daten der Reklamationen beziehen sich nur auf so gravierende Fehler, dass die NCAs komplett ausgefallen sind. Für die beim Kunden aufgetretenen Funktionsstörungen, die bei diesem behoben werden konnten, stehen keine Daten zur Verfügung. Eine weitergehende Bewertung ist deswegen auch hier nicht möglich und es können nur die einzelnen bekannt gewordenen Funktionsstörungen aufgeführt und in einen Zusammenhang mit deren möglichen Ursachen gestellt werden.
                                            




Besonders die aus dem Reklamationsmanagement aufgeführten Funktionsstörungen (vgl. Kapitel~\ref{cha:Auswertung_des_Reklamationsmanagements}) sind nicht immer einer eindeutigen Ursache zuzuordnen und es können auch gleichzeitig mehrere Ursachen zugrunde liegen.

%Fressen bzw. vorher  Beim Verkratzen bzw. sogar Fressen er Laufflächen der Pinole. Eine mögliche Ursache ist hier mangelnde Schmierung. Es ist bekannt, dass eine außermittige Belastung dieses Problem verschärft, genauso wie sehr kurze Verfahrbewegungen. so dass nicht der ganze weg verfahren wird


Das Fressen oder vorherige Verkratzen der Lauffläche der Pinole ist eine massive Funktionsstörung. Dafür kommt eine mangelnde Schmierung als Ursache in Frage. Es ist bekannt, dass eine außermittige Belastung dieses Problem verschärft, genauso wie sehr kurze Verfahrbewegungen. Warum der Wellendichtring an der Pinole des Öfteren dazu neigt, kaputt zu gehen, ist nicht genau zu klären. Auch die im Reklamationsmanagementsystem erfassten Fehler, wie ein allgemeiner Fehler am AMO-Messsystem, die Undichtigkeit am Kühlsystem des Motors oder der Motordefekt, bei dem der Fehler beim Lieferanten liegt, können nicht eindeutig einer Ursache zugeordnet werden und bedürfen genauerer Untersuchungen.


Der konstruktive Fehler, dass Schmieröl in den Motor und anschließend in den Motorgeber gelangen kann, wurde bereits durch konstruktive Maßnahmen beseitigt. Allerdings sind noch einige Achsen mit dieser Schwachstelle im Umlauf. So ist in der nächsten Zeit mit weiteren Reklamationen aufgrund dieser Ursache zu rechnen. Ein konstruktiver Fehler ist weiterhin, dass der Kleber für das Messlineal nicht für die ölhaltige Umgebung und die hohen Temperaturen vorgesehen ist.


 Fertigungsfehler der Einzelkomponenten treten im Produktionsprozess regelmäßig auf. Als Ursache dafür ist anzusehen, dass Bauteile nicht zeichnungsgemäß gefertigt sind und nachgearbeitet oder ausgetauscht werden müssen. Dies führt zu erheblichen Verzögerungen während der Montage. In der Montage kann es zu Fehlern kommen, weil die Abstimmscheiben nicht richtig abgestimmt sind, was zu dem erheblichen Problem einer zu großen Vorspannung der Lager oder von axialem Spiel kommt. Axiales Spiel ergibt sich auch durch nicht richtiges Anziehen der Spannmutter in der Pinole. Durch Abändern der Reihenfolge bei der Montage kann das falsche Anziehen der Spannmutter verhindert werden.





Beispiele für eine fehlerhafte Programmierung als Ursache von Funktionsstörungen treten z.B. in Form falsch programmierter Parameter immer wieder auf. Denn beim Einrichten eines neuen Werkzeuges müssen alle Parameter für das Aggregat manuell eingegeben werden. Dies führt zu unerwarteten Fehlern und längeren Fehlersuchen.


Ein Grund zur Reklamation kann in der Überlastung des Motors und ein anderer in der Beschädigung des Gleitsteins gesehen werden. Dabei  wird aufgrund von falscher Handhabung oder falscher Programmierung auf Block gefahren. Beide Male wird von  nicht bestimmungsgemäßem Gebrauch der NCAs ausgegangen.

Ein zugesetzter Kühlkreislauf stört die Funktion erheblich. Er kann als Beispiel für die nicht bestimmungsgemäße Instandhaltung, die eine Funktionsstörung verursacht, angesehen werden. Dies ist insbesondere ein Problem bei unsachgemäßem Umgang mit dem Kühlmittel. Auch durch Späne oder sonstige Gegenstände in den Kühlkanälen ist der Durchfluss nicht sichergestellt.


Verschleiß und Alterungserscheinungen sind als Fehlerquelle nach längerem Gebrauch der NCAs zu erwarten. So fallen die Motoren z.T. nach längerem Gebrauch aus und auch das Fressen der Pinole könnte als Verschleiß gewertet werden.


Ein Überblick über die bekannten Fehler und ihre Ursachen findet sich im Anhang~\ref{cha:Uebersicht_ueber_aufgetretene_Fehler}.








\begin{comment}
\begin{itemize}
 \item konstruktive Fehler
 \begin{itemize}
    \item Kleber für Messlineal ist nicht für die ölhaltige Umgebung und Temperaturen vorgesehen
    \item Der konstruktive Fehler, dass Schmieröl in den Motor und anschließend in den Motorgeber gelangen kann, wurde bereits durch konstruktive Maßnahmen beseitigt. Allerdings sind noch einige Achsen mit dieser Schwachstelle im Umlauf. So ist in der nächsten Zeit mit weiteren Reklamationen aufgrund dieser Ursache zu rechnen.
 \end{itemize}
 
 
 
 
 
 \item Fertigungsfehler der Einzelkomponenten
 \begin{itemize}
    \item Immer wieder müssen nicht zeichnungsgemäß gefertigte Bauteile nachgearbeitet oder ausgetauscht werden. Dies führt zu erheblichen Verzögerungen während der Montage.
 \end{itemize}
 
 
 
 \item Fehler bei der Montage 
 \begin{itemize}
    \item Durch Fehler bei der Montage ergibt sich axiales Spiel, insbesondere durch nicht richtiges Anziehen der Spannmutter in der Pinole. Durch abändern der Montage Reihenfolge kann das falsche Anziehen der Spannmutter verhindert werden. Durch falsches Abstimmen der Abstimmscheibe kann es allerdings immer noch zu axialem Spiel kommen.
    \item Durch nicht richtig abgestimmte Abstimmscheiben kann es entweder zu axialem Spiel oder einer zu großen Vorspannung der Lager kommen.
 \end{itemize}
 
 
 
 
 \item fehlerhafte Programmierung
 \begin{itemize}
    \item Falsch programmierte Parameter treten immer wieder auf, da beim Einrichten eines neuen Werkzeuges alle Parameter für das Aggregat manuell eingegeben werden müssen. Dies führt immer wieder zu unerwarteten Fehlern und längeren Fehlersuchen.
 \end{itemize}
 
 
 \item nicht bestimmungsgemäßer Gebrauch
 \begin{itemize}
    \item Überlastung des Motors. Reklamation
    \item Gleitstein beschädigt (auf Block gefahren). Falsche Handhabung oder falsche Programmierung
 \end{itemize}
 
 
 
 
 \item nicht bestimmungsgemäße Instandhaltung
 \begin{itemize}
    \item Kühlkreislauf ist zugesetzt. Dies ist insbesondere ein Problem bei unsachgemäßigen Umgang mit dem Kühlschmiermittel. Auch durch Späne oder sonstige Gegenstände in den Kühlkanalen ist der Durchfluss nicht sichergestellt.
 \end{itemize}
 
 
 \item Verschleiß und Alterungserscheinungen
 
 \item nicht eindeutig zuordenbare Fehler
 \begin{itemize}
    \item Fressen bzw. vorher Verkratzen der Laufflächen der Pinole. Eine mögliche Ursache ist hier mangelnde Schmierung. Es ist bekannt, dass eine außermittige Belastung dieses Problem verschärft, genauso wie sehr kurze Verfahrbewegungen. so dass nicht der ganze Weg verfahren wird
    \item allgemeiner Fehler AMO-Messsystem (Reklamationsmanagementsystem)
    \item Defekt am Motor (Fehler b. Lieferanten) (Reklamationsmanagementsystem)
    \item Der Wellendichtring an der Pinole neigt des öfteren dazu, kaputt zu gehen oder nicht die geforderte Dichtigkeit zu erreichen.
    \item Motor: Kühlsystem undicht (Reklamationsmanagementsystem)
    \item Haltebremse versagt
 \end{itemize}
\end{itemize}


\end{comment}



















