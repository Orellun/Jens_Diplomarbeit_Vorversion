

\begin{tikzpicture}
 \begin{axis}[
 	no markers,
 	ymin=0,
 	%ymax=60,
    xmin=0,
    xmax=162,
 	width=\textwidth,height=0.3\textheight,
  	%title=Einfahrzyklus Drehmomentprofil,
    xlabel={Frequenz in \si{\hertz}},
    ylabel={Beschleunigung in mg},
    grid=major,
    %legend entries={Temperatursensor 1,Temperatursensor 2,Temperatursensor 3},
    %legend pos=south east,
    %enlarge x limits=0.01,
]
 	\addplot table[x=Hertz, y=mg]  {graphen/CSV_Daten/Schwingungsmessung_200mm_Seonsor1_HFFT_0,2_0-160.txt};
 	\draw(current axis.south-|{axis cs:76,1})%
       --(current axis.north-|{axis cs:76,1})node[left,pos=0.7]{Wälzkörper 76 Hz};
    \draw(current axis.south-|{axis cs:89,1})%
       --(current axis.north-|{axis cs:89,1})node[right,pos=0.7]{89 Hz Aussenring};
    \draw(current axis.south-|{axis cs:131,1})%
       --(current axis.north-|{axis cs:131,1})node[right,pos=0.7,align=left]{131 Hz \\ Innenring};
 \end{axis}
\end{tikzpicture}